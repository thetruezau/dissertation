\chapter*{Abstract}
\thispagestyle{title_on_header}

This dissertation presents the mechanisation within the \textit{Rocq Prover} of a multiary $\Lam$-calculus (system $\LamM$) and associated metatheory.
This mechanisation has two primary objectives: first, a formal verification of metatheoretical proofs, which are often lengthy and prone to human error in manual treatments; second, a study of system $\LamM$ as an extension of the Curry-Howard isomorphism to the sequent calculus paradigm.
Our development uses a de~Bruijn representation to mechanise syntax with binders and uses the \textit{Autosubst} library to define the desired capture-avoiding substitution operations.
The formalisation of the system $\LamM$ includes its reduction rules, typing system and core theorems like subject reduction.
A major contribution is the study of the canonical subsystem within $\LamM$, an isomorphism of this subsystem with the simply typed $\Lam$-calculus and a conservativeness result.
Interestingly, by formalising the isomorphism with the $\Lam$-calculus, we were able to derive the confluence for $\LamM$.
% This work details a novel mechanisation of this metatheory and describes the rigorous methodology implemented.
Throughout this dissertation we offer a detailed description of this novel mechanisation of system $\LamM$ and of its metatheory, including commentary on methodological aspects.

\paragraph{Keywords} $\lambda$-calculus, sequent calculus, metatheory, \textit{Rocq Prover}, \textit{Autosubst}

%%% Local Variables:
%%% mode: LaTeX
%%% TeX-master: "../dissertation"
%%% End:
