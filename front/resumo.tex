\chapter*{Resumo}
\thispagestyle{title_on_header}

Esta dissertação apresenta a mecanização no \textit{Rocq Prover} de um cálculo-$\Lam$ multiário (sistema $\LamM$) e da sua metateoria associada.
Esta mecanização tem dois objectivos principais: primeiro, uma verificação formal de provas acerca de metateoria, que são frequentemente longas e propensas a erros humanos nos tratamentos manuais; segundo, um estudo do sistema $\LamM$ como uma extensão do isomorfismo de Curry-Howard para o paradigma do cálculo de sequentes.
O nosso desenvolvimento usa uma representação de~Bruijn para mecanizar a sintaxe com mecanismos de ligação e usa a biblioteca \textit{Autosubst} para definir as operações de substituição sem captura de variáveis.
A formalização do sistema $\LamM$ inclui as suas regras de redução, o sistema de tipificação e teoremas centrais como a redução do sujeito.
Uma contribuição importante é o estudo do subsistema canónico em $\LamM$, um isomorfismo deste subsistema com o cálculo-$\Lam$ com tipos simples e um resultado de conservatividade.
Curiosamente, ao formalizar o isomorfismo com o cálculo-$\Lam$, conseguimos derivar a confluência para $\LamM$.
Ao longo desta dissertação oferecemos uma descrição detalhada desta nova mecanização do sistema $\LamM$ e da sua metateoria, incluindo comentários sobre aspectos metodológicos.

\paragraph{Palavras-chave} cálculo-$\lambda$, cálculo de sequentes, \textit{Rocq Prover}, \textit{Autosubst}

%%% Local Variables:
%%% mode: LaTeX
%%% TeX-master: "../dissertation"
%%% End:
