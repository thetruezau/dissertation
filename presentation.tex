\documentclass{beamer}
\usepackage[utf8]{inputenc}

\usepackage{ebproof}
\usepackage{comment}
\usepackage{graphicx}

\let\oldfootnote\footnote
\renewcommand{\footnote}{\only<+->\oldfootnote}

\usepackage{palatino}
\usepackage{mathpazo}

\usetheme{Madrid}
\usecolortheme{default}

%------------------------------------------------------------
%This block of code defines the information to appear in the
%Title page
\title[Dissertation Defense] %optional
{On the mechanisation of the multiary lambda calculus and subsystems}

\author[Miguel Alves]
{Miguel Alves}

\institute[UMinho]{Escola de Ciências \\ Universidade do Minho}

\date{\today}

\AtBeginSection[]
{
	\begin{frame}
		\frametitle{Table of Contents}
		\tableofcontents[currentsection]
	\end{frame}
}

\begin{document}

\frame{\titlepage}

% ----
% SEC1
% ----
\section{Motivation and roadmap}

\begin{frame}
	\frametitle{Why mechanise?}
\end{frame}

\begin{frame}
	\frametitle{Why the multiary $\lambda$-calculus?}
\end{frame}

% ----
% SEC2
% ----
\section{Background}
\begin{frame}
	\frametitle{Simply typed $\lambda$-calculus}
\end{frame}

\begin{frame}
	\frametitle{$\lambda$-calculus with de Bruijn syntax}
\end{frame}

\begin{frame}
	\frametitle{The \textit{Rocq Prover}}
\end{frame}	

% ----
% SEC3
% ----
\section{Multiary $\lambda$-calculus and subsystems}
\begin{frame}
	\frametitle{System $\lambda m$}
\end{frame}

\begin{frame}
	\frametitle{Canonical subsystem $\lambda m^{Can}$}	
\end{frame}

% ----
% SEC4
% ----
\section{Canonical $\lambda$-calculus}
\begin{frame}
	\frametitle{System $\vec \lambda$}
\end{frame}

\begin{frame}
	\frametitle{$\vec \lambda$ vs $\lambda m^{Can}$}
\end{frame}

\begin{frame}
	\frametitle{Conservativeness}	
\end{frame}

% ----
% SEC5
% ----
\section{An isomorphism with the simply typed $\lambda$-calculus}
\begin{frame}
	\frametitle{Mappings $\theta$ and $\psi$}
\end{frame}


% ----
% SEC6
% ----
\section{Conclusions and discussion}
\begin{frame}
	\frametitle{\textit{Autosubst} library}
\end{frame}

% ----
% SEC7
% ----
\section{A showcase of the mechanisation in the \textit{Rocq Prover}}
\begin{frame}
	\frametitle{\textit{Autosubst} library}
\end{frame}

\begin{frame}
	\frametitle{Conservativeness result?}
\end{frame}

\end{document}