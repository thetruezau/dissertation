\documentclass{beamer}
\setbeamercovered{transparent}
% \usepackage[utf8]{inputenc}

% \setlength{\parindent}{0em}

\usepackage{ebproof}
\usepackage{tikz-cd}
\usepackage{graphicx}
% \usepackage{palatino}
% \usepackage{fontspec}
\usepackage{mathpazo}
\usepackage{xcolor}
\usepackage{color}
\usepackage{caption}
\usepackage{cancel}

\let\oldfootnote\footnote
\renewcommand{\footnote}{\only<+->\oldfootnote}

% \newtheorem{proposition}{Proposition}
% \crefname{proposition}{Proposition}{Propositions}

\newcommand{\LamM}{\pmb{\lambda m}}
\newcommand{\LamVM}{\pmb{\vec \lambda m}}

\newcommand{\Lam}{\pmb{\lambda}}
\newcommand{\LamV}{\pmb{\vec \lambda}}

\newcommand{\LamCan}{\LamM^{Can}}


\usetheme{Madrid}
\usecolortheme{default}

% ------------------------------------------------------------
% This block of code defines the information to appear in the
% Title page
\title[Dissertation Defense] % optional
{On the mechanisation of the multiary lambda calculus and subsystems}

\author[Miguel Alves]
{Miguel Alves}

\institute[UMinho]{Escola de Ciências \\ Universidade do Minho}

\date{\today}

\AtBeginSection[]
{
  \begin{frame}
    \frametitle{Table of Contents}
    \tableofcontents[currentsection]
  \end{frame}
}

\begin{document}

\frame{\titlepage}

% ---
% SEC1
% ---

\section{Multiary $\Lam$-calculus and subsystems}
\subsection{Multiary $\Lam$-calculus}
\begin{frame}
  \frametitle{From $\Lam$ to $\LamM$}
  
  \only<3->{$\Lam$-calculus with a multiary application!}
  \begin{definition}[Expressions] \vspace{-15pt}
    \begin{align*}
      \only<1-2>{ (\Lam \text{-terms}) && \hspace{-1cm} t, u, v &::= \ x \ | \ \lambda x . t \ | \ \only<1>{t u}
                                                                                                   \only<2>{\xcancel{t u}} }
      \only<3->{ (\LamM \text{-terms}) && \hspace{-1cm} t, u, v &::= \ x \ | \ \lambda x . t \ | \ t(u, l) }
      \only<4->{ \\ (\LamM \text{-lists}) && \hspace{-2cm} l &::= \ []\  | \ u :: l }
    \end{align*}
  \end{definition}

  \begin{definition}[Reduction rules]<1,5-> \vspace{-15pt}
    \begin{align*}
      \only<1-4>{ (\beta) & \qquad (\lambda x . t) u \to t[x := u] }
      \only<5->{ (\beta_1) & \qquad (\lambda x . t)(u, []) \to t[x := u] }
      \only<6->{ \\ (\beta_2) & \qquad (\lambda x . t)(u, v::l) \to t[x := u](v, l) }
      \only<7->{ \\ (h) & \qquad t(u, l)(u', l') \to t(u, l + (u'::l')) }
    \end{align*}
  \end{definition}
  
  \begin{block}{Typing rules}<1,8->
    \only<1-7>{ Typing rules are isomorphic to natural deduction (Curry-Howard isomorphism). }
    \only<8->{ Typing rules are isomorphic to a fragment of the sequent calculus (similar to system $LJT$). }
  \end{block}
\end{frame}

\subsection{Canonical subsystem}
\begin{frame}
  \frametitle{$h$-normal forms}
  The purpose of the $h$ reduction rule
  \[ (h) \qquad {\color{red} t(u, l)}(u', l') \to {\color{red} t}(u, l + (u'::l')) \]
  is to reorganise our terms, bringing either a variable or an abstraction to the head of applicative terms.

  \onslide<2->{ We will identify the subsets $Can$ (canonical terms) and $CanList$ (canonical lists) of $h$-nfs, having applicative terms restricted to \[ x(u, l) \quad \text{ or } \quad (\lambda x . t)(u, l). \] }
  \onslide<3->{ Moreover, we easily define a map \[ h: \LamM \text{-terms} \to Can. \] }
  % collapsing $\Lam$-terms to their $h$-normal form.
  \onslide<4->{ This map suffices to induce notions of reduction and typification over these subexpressions, creating what we will call a \underline{canonical subsystem}. }
\end{frame}

\subsection{Canonical $\Lam$-calculus}
\begin{frame}
  \frametitle{From $\LamM$ to $\LamV$}
  
  \only<4->{ Canonical applicative terms are split in two constructors! }
  \begin{definition}[Expressions] \vspace{-15pt}
    \begin{align*}
      \only<1-2>{ (\LamM \text{-terms}) && \hspace{-1cm} t, u, v &::= \ x \ | \ \lambda x . t \
                                                                   | \ \only<1>{ t(u, l) } \only<2>{\xcancel{ t(u, l) }} \\ }
      \only<3>{ (\LamV \text{-terms}) && \hspace{-1cm} t, u, v &::= \ x \ | \ \lambda x . t \ | \ app_v(x, u, l) \\ }
      \only<4->{ (\LamV \text{-terms}) && \hspace{-1cm} t, u, v &::= \ x \ | \ \lambda x . t \ | \ app_v(x, u, l) \ | \ app_\lambda(x.t, u, l) \\ }
      (\only<1-2>{ \LamM } \only<3->{ \LamV } \text{-lists}) && \hspace{-2cm} l &::= \ [] \  | \ u :: l
    \end{align*}
  \end{definition}

  \begin{definition}[Reduction rules]<1,5-> \vspace{-15pt}
    \begin{align*}
      \only<1-4>{(\beta_1) & \qquad (\lambda x . t)(u, []) \to t[x := u] \\ }
      \only<5->{(\vec \beta_1) & \qquad app_\lambda (x . t, u, []) \to t[x := u] \\ }
      \only<1-4>{ (\beta_2) & \qquad (\lambda x . t)(u, v::l) \to t[x := u](v, l) }
      \only<6-7>{ (\vec \beta_2) & \qquad app_\lambda (x . t, u, v::l) \to t[x := u](v, l) }
      \only<8->{ (\vec \beta_2) & \qquad app_\lambda (x . t, u, v::l) \to t[x := u]@(v, l) }
      \only<1-4>{ \\ (h) & \qquad t(u, l)(u', l') \to t(u, l + (u'::l')) }
      \only<7>{ \\ \xcancel{ (h) } & \qquad t(u, l)(u', l') \to t(u, l + (u'::l')) }
    \end{align*}
  \end{definition}

  \only<8->{ *In the $(\vec \beta_2)$ rule, the $@$ operator inspects the first term $t[x:=u]$ in order to produce a new applicative term (a generic application $t(u, l)$ is not part of this new syntax). }
\end{frame}

\begin{frame}
  \frametitle{Typing system of $\LamV$}
  \begin{definition}[Typing rules for $\LamV$-expressions]
    \[
      \begin{prooftree}
        \infer0[Ax]{ x:A, \Gamma \vdash var(x):A } 
      \end{prooftree}
      \qquad
      \begin{prooftree}
        \hypo{ x:A, \Gamma \vdash t:B }
        \infer1[Right]{ \Gamma \vdash \lambda x . t : A \supset B  } 
      \end{prooftree}
    \]
    \[
      \begin{prooftree}
        \hypo{ \Gamma' \vdash u:A}
        \hypo{ \Gamma' ; B \vdash l:C }	
        \infer2[Left]{ \underbrace{ \Gamma, x: A \supset B }_{\Gamma'} \vdash app_v (x, u, l) : C } 
      \end{prooftree}
    \]
    \[
      \begin{prooftree}
        \hypo{ \Gamma \vdash u:A }
        \hypo{ \Gamma, x:A \vdash t:B }
        \hypo{ \Gamma ; B \vdash l : C }	
        \infer3[DoubleCut]{ \Gamma \vdash app_\lambda (x . t, u, l) : C } 
      \end{prooftree}
    \]
    \[
      \begin{prooftree}
        \infer0[Ax*]{ \Gamma ; A \vdash []:A } 
      \end{prooftree}
      \qquad
      \begin{prooftree}
        \hypo{ \Gamma \vdash u:A }
        \hypo{ \Gamma ; B \vdash l:C }
        \infer2[Left*]{ \Gamma ; A \supset B \vdash  u::l : C } 
      \end{prooftree}
    \]
  \end{definition}
\end{frame}

\begin{frame}
  \frametitle{$\LamCan$ vs $\LamV$}  
  % We prove that two different definitions for the canonical subsystem result in isomorphic systems.

  % https://q.uiver.app/#q=WzAsNyxbMCwwLCJcXGJveGVke1xccG1ie1xcbGFtYmRhIG19fSJdLFswLDEsIlxcYm94ZWR7XFxwbWJ7XFxsYW1iZGEgbV57Q2FufX19Il0sWzIsMSwiXFxib3hlZHtcXHBtYntcXHZlYyBcXGxhbWJkYX19Il0sWzAsMywiXFxMYXJnZXtcXGZyYWN7dCBcXCBcXHRvX1xcYmV0YSBcXCB0JyBcXCAoXFx0ZXh0e2luIH1cXGxhbWJkYSBtKX17aCh0KSBcXCAoXFx0b19cXGJldGEpX2MgXFwgaCh0Jyl9fSJdLFsyLDMsInQgXFwgXFx0b197XFx2ZWMgXFxiZXRhfSBcXCB0JyJdLFswLDIsIihcXGxhbWJkYSB4LnQpKHUsIGwpIl0sWzIsMiwiYXBwX1xcbGFtYmRhICh4LnQsIHUsIGwpIl0sWzAsMSwiaCIsMl0sWzAsMiwicCJdLFsyLDEsImkiXV0=
  \[
    \begin{tikzcd}[sep=small]
	{\boxed{\LamM}} \\
	{\boxed{\LamCan}} && {\boxed{\LamV}} \\
	{(\lambda x.t)(u, l)} && {app_\lambda (x.t, u, l)} \\
	{\Large{\frac{t \ \to_\beta \ t' \ (\text{in }\lambda m)}{h(t) \ (\to_\beta)_c \ h(t')}}} && {t \ \to_{\vec \beta} \ t'}
	\arrow["h \ "', from=1-1, to=2-1]
	\arrow["p", from=1-1, to=2-2]
	\arrow["i", from=2-2, to=2-1]
      \end{tikzcd}
    \]

    \vspace{-5pt}
    \begin{theorem}[Isomorphism of reduction]<2-> \hfill
      \begin{enumerate}
      \item $t \to_{\vec \beta} t'$ in $\LamV$ $\iff$ $i(t) \ (\to_{\beta})_c \ i(t')$ in $\LamCan$
      \item $t \ (\to_{\beta})_c \ t'$ in $\LamCan$ $\iff$ $p(t) \to_{\vec \beta} p(t')$ in $\LamV$
      \end{enumerate}
    \end{theorem}
\end{frame}

\begin{frame}
  \frametitle{Conservativeness}
  \begin{theorem}[Conservativeness]
    For every $\LamV$-terms $t$ and $t'$, we have:
    \[ t \twoheadrightarrow_{\vec \beta} t' \iff i(t) \twoheadrightarrow_{\beta h} i(t'). \]
  \end{theorem}

  \begin{corollary}[$\LamM$ is a conservative extension of $\LamCan$]<2->
    % In other words,
    For every $t, t' \in Can$, 
    \[ t \twoheadrightarrow_{\beta^{Can}} t' \iff t \twoheadrightarrow_{\beta h} t', \]
    where $\twoheadrightarrow_{\beta^{Can}}$ is the reflexive-transitive closure of $(\to_{\beta})_c$.
  \end{corollary}
\end{frame}

\begin{frame}
  \frametitle{$\LamV$ vs $\Lam$}

  % https://q.uiver.app/#q=WzAsNixbMCwwLCJcXGJveGVke1xccG1ie1xcdmVjIFxcbGFtYmRhfX0iXSxbMCwxLCJhcHBfXFxsYW1iZGEgKHgudCwgdSwgW3ZfMSwgdl8yLCBcXGRvdHMsIHZfa10pIl0sWzIsMCwiXFxib3hlZHtcXHBtYntcXGxhbWJkYX19Il0sWzIsMSwiKCgoKChcXGxhbWJkYSB4LnQpIHUpIHZfMSkgdl8yKSBcXGRvdHMpIHZfayJdLFswLDIsInQgXFwgXFx0b197XFx2ZWMgXFxiZXRhfSBcXCB0JyJdLFsyLDIsInQgXFwgXFx0b197XFxiZXRhfSBcXCB0JyJdLFswLDIsIlxcdGhldGEiLDAseyJvZmZzZXQiOi0xfV0sWzIsMCwiXFxwc2kiLDAseyJvZmZzZXQiOi0xfV1d
  \[\begin{tikzcd}[sep=small]
      {\boxed{\LamV}} && {\boxed{\Lam}} \\
      {app_v (x, u, [v_1, v_2, \dots, v_k])} && {((((x u) v_1) v_2) \dots) v_k} \\
      {app_\lambda (x.t, u, [v_1, v_2, \dots, v_k])} && {(((((\lambda x.t) u) v_1) v_2) \dots) v_k} \\
      {t \ \to_{\vec \beta} \ t'} && {t \ \to_{\beta} \ t'}
      \arrow["\theta", shift left, from=1-1, to=1-2]
      \arrow["\psi", shift left, from=1-2, to=1-1]
    \end{tikzcd}\]
  
  \begin{theorem}[Isomorphism of reduction]<2-> \hfill
    \begin{enumerate}
    \item $t \to_{\vec \beta} t' \text{ in } \LamV \iff \theta(t) \to_{\beta} \theta(t') \text{ in } \Lam$
    \item $t \to_{\beta} t' \text{ in } \Lam \iff \psi(t) \to_{\vec \beta} \psi(t') \text{ in } \LamV$
    \end{enumerate}
  \end{theorem}
\end{frame}

% ----
% SEC2
% ----
\section{On the mechanisation}
\subsection{Key choices on the mechanisation}
\begin{frame}
  \frametitle{Our key choices on the mechanisation}
  \begin{itemize}
  \item<1-> \underline{Proof assistant}: \textit{Rocq Prover}
  \item<2-> \underline{Deal with binders}: de Bruijn indices
    
    \only<3->{ Example: a $\Lam$-term $\lambda x. \lambda y . y$ is represented as $\lambda . \lambda . 0$ }

  \item<4-> \underline{Libraries}: \textit{Autosubst} library to define substitution operations using de Bruijn indices and parallel substitutions
  \end{itemize}
\end{frame}

\subsection{\textit{Autosubst} library}
\begin{frame}
  \frametitle{A main difference: parallel substitutions}
  We are used to deal with a substitution operation that replaces one variable at a time --- $t[x:=u]$.

  \begin{itemize}
  \item<2-> Using a combination of de Bruijn indices and parallel substitutions (that replace every free variable at a time) we can simplify our representations and definitions. 
  \item<3-> Parallel substitutions can be seen as sequences, $\sigma = (t_0, t_1, t_2, \dots)$, and the operation $t[\sigma]$ replaces with $t_0$ the variable index $0$, with $t_1$ the variable index $1$ and so on.
  \item<4-> We also decided to simplify the presentation of our typing systems using contexts that are substitutions over types, $\Gamma = (A, B, C, \dots)$.
    % \subitem This yiels a nonstandard aproach to typing systems using infinite contexts...
  \end{itemize}
\end{frame}

\begin{frame}
  \frametitle{$\LamM$ vs its ``mechanised self''}

  \renewcommand{\arraystretch}{5} % Default value: 1
  \[ \begin{matrix}    
    t, u, v ::= \ x \ | \ \lambda x . t \ | \ t(u, l)
    &
      \begin{figure}[h]
        \centering
        \includegraphics[width=0.5\textwidth]{pictures/screenshot_terms}
      \end{figure}
    \\
    \onslide<2->{
    (\beta_1) \quad (\lambda x . t)(u, []) \to t[x := u]
    &
      \begin{figure}[h]
        \centering
        \includegraphics[width=0.5\textwidth]{pictures/screenshot_beta1}
      \end{figure}
      }
    \\
    \onslide<3->{
    \begin{prooftree}
      \hypo{ x:A, \Gamma \vdash t:B }
      \infer1[Right]{ \Gamma \vdash \lambda x . t : A \supset B  }
    \end{prooftree}
    &
      \begin{figure}[h]
        \centering
        \includegraphics[width=0.5\textwidth]{pictures/screenshot_typing}
      \end{figure}
      }
  \end{matrix} \]
\end{frame}

\subsection{Hands on the proof assistant}
\begin{frame}
  \frametitle{Hands on the proof assistant}
    \begin{theorem}[Conservativeness]
    For every $\LamV$-terms $t$ and $t'$, we have:
    \[ t \twoheadrightarrow_{\vec \beta} t' \iff i(t) \twoheadrightarrow_{\beta h} i(t'). \]
  \end{theorem}
\end{frame}

% ---
% SEC 3
% ---

\section{Take home}
\begin{frame}
  \frametitle{Take home messages}
  \begin{itemize}
  \item<1-> A simple, yet, computationally interesting idea like changing the syntax of the $\Lam$-calculus application has a logical meaning involving sequent calculus.
  \item<2-> A mechanisation cannot hide details and may lead to a different presentation (when comparing to the paper one).
    % \item<3-> Mechanisation works can help clarifying
  \item<3-> Mechanising metatheory can be boosted by using the available libraries (\textit{Autosubst} in our case) and generalisation of many concepts is a good path to code reusability.
  \end{itemize}
\end{frame}

\end{document}
