\documentclass{beamer}
% \usepackage[utf8]{inputenc}

\usepackage{ebproof}
\usepackage{comment}
\usepackage{graphicx}
% \usepackage{palatino}
\usepackage{fontspec}
\usepackage{mathpazo}
\usepackage{xcolor}
\usepackage{color}
\usepackage{caption}


\usepackage{cancel}

\let\oldfootnote\footnote
\renewcommand{\footnote}{\only<+->\oldfootnote}


\newcommand{\LamM}{\pmb{\lambda m}}
\newcommand{\LamVM}{\pmb{\vec \lambda m}}

\newcommand{\Lam}{\pmb{\lambda}}
\newcommand{\LamV}{\pmb{\vec \lambda}}

\newcommand{\LamCan}{\LamM^{Can}}


\usetheme{Madrid}
\usecolortheme{default}

%------------------------------------------------------------
%This block of code defines the information to appear in the
%Title page
\title[Dissertation Defense] % optional
{On the mechanisation of the multiary lambda calculus and subsystems}

\author[Miguel Alves]
{Miguel Alves}

\institute[UMinho]{Escola de Ciências \\ Universidade do Minho}

\date{\today}

\AtBeginSection[]
{
  \begin{frame}
    \frametitle{Table of Contents}
    \tableofcontents[currentsection]
  \end{frame}
}

\begin{document}

\frame{\titlepage}

% ---
% SEC1
% ---

\section{Multiary $\Lam$-calculus and subsystems}
\subsection{System $\LamM$}
\begin{frame}
  \frametitle{System $\LamM$}
  
  \onslide<3->{A $\Lam$-calculus with multiary applications!}
  \begin{definition}[Expressions]
    \begin{align*}
      \only<1-2>{ (\Lam \text{-terms}) && \hspace{-1cm} t, u, v &::= \ x \ | \ \lambda x . t \ | \ \only<1>{t u}
                                                                                                   \only<2>{\xcancel{t u}} }
      \only<3->{ (\LamM \text{-terms}) && \hspace{-1cm} t, u, v &::= \ x \ | \ \lambda x . t \ | \ t(u, l) }
      \only<4->{ \\ (\LamM \text{-lists}) && \hspace{-2cm} l &::= \ []\  | \ u :: l }
    \end{align*}
  \end{definition}

  \begin{definition}[Reduction rules]
    \begin{align*}
      \only<1-4>{ (\beta) & \qquad (\lambda x . t) u  \to t[x := u] }
      \only<5->{ (\beta_1) & \qquad (\lambda x . t)(u, [])   \to t[x := u] }
      \only<6->{ \\ (\beta_2) & \qquad (\lambda x . t)(u, v::l) \to t[x := u](v, l) }
      \only<7->{ \\ (h)       & \qquad t(u, l)(u', l')          \to t(u, l + (u'::l')) }
    \end{align*}
  \end{definition}
\end{frame}

\subsection{Canonical subsystem $\LamCan$}

\begin{frame}
  \frametitle{$h$-normal forms}
  The purpose of the $h$ reduction rule
  \[ (h) \qquad t(u, l)(u', l') \to t(u, l + (u'::l')) \]
  is to bring either a variable or an abstraction to the head of applications.

  \vspace{15pt}
  \onslide<2->{
    We can identify such subset of expressions, having only applications
    \[ x(u, l) \quad \text{ or } \quad (\lambda x . t)(u, l). \]
  }
\end{frame}

\subsection{Canonical $\Lam$-calculus}
\begin{frame}
  \frametitle{System $\LamV$}
\end{frame}

\begin{frame}
  \frametitle{$\LamV$ vs $\LamCan$}
\end{frame}

\begin{frame}
  \frametitle{Conservativeness}	
\end{frame}

\begin{frame}
  \frametitle{$\LamV$ vs $\Lam$}
\end{frame}

% ----
% SEC2
% ----
\section{On the mechanisation}
\subsection{\textit{Our main choices on mechanisation}}
\begin{frame}
  \frametitle{Our main choices on mechanisation}
  \begin{itemize}
  \item<1-> Proof assistant: \textit{Rocq Prover}
  \item<2-> Deal with binders: de Bruijn indices
  \item<3-> Dependencies on libraries: \textit{Autosubst} library to define substitution operations using de Bruijn indices and parallel substitutions
  \end{itemize}
\end{frame}

\subsection{\textit{Autosubst} library}
\begin{frame}
  \frametitle{$\sigma$-calculus and parallel substitutions}
  
\end{frame}

\begin{frame}
  \frametitle{System $\LamM$ using \textit{Autosubst}}
\end{frame}

\subsection{A showcase}
\begin{frame}
  \frametitle{Mechanisation of the conservativeness result}
\end{frame}

\end{document}
