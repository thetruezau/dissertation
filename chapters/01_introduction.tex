\chapter{Introduction}
\label{c:intro}

\section{Motivation}
Taking the title for this dissertation, we can explain what motivated the following dissertation by asking ourselves: ``Why mechanise metatheory?'' and ``Why the multiary $\Lam$-calculus?''.

Before addressing these questions formally, we could just say that mechanising mathematics is an enjoyable task.
And for what matters, this could be all we say about our motivation.
All I intend to say is that even if our work in mathematics had no application or direct consequences, the fun of doing it would be a good enough motivation.
Mechanising mathematics is like a computer game for a mathematician.

\subsection{Why mechanise metatheory?}
The mechanisation of metatheory has been taken seriously for at least 20 years~\cite{POPLmark}.
By mechanisation we mean a well-founded description of a mathematical object using a proof assistant.
Such proof assistants have attracted the attention of mathematicians because of the reliability and automation they provide for writing computer verified proofs~\cite{FourColourThm}.
There has also been an increasing interest by engineers in the use of such tools for the security guarantees achieved when formally proving properties about computer programmes~\cite{CompCert}.

One could even argue that any work of mechanisation is useful, because it will:
\begin{enumerate}
\item result in a computer verified work,
\item expose the difficulties behind any mathematical formalisation,
\item provide automation for routine and tedious parts,
\item potentially allow some theory to be extended with less cost.
\end{enumerate}

All of the latter are perfectly good motivations for the work done in this dissertation.

Then, the question of \textit{Why metatheory?} arises.
From the reasons above, some are highlighted by the task of mechanising metatheory.
For example, it is often argued that metatheoretical proofs \textit{``are long, contain few essential insights, and have a lot of tedious but error-prone cases''}~\cite{AutosubstSchafer}.

In our case, mechanising theoretical results related to the multiary $\Lam$-calculus could enable new ways of continuing the work being done in this topic.
Curiously, our work with an unusual version of the $\Lam$-calculus could even suggest some improvements in already mature tools for mechanising metatheory (as is the case with the used \textit{Autosusbt} library for the \textit{Rocq Prover}).

\subsection{Why the multiary $\Lam$-calculus?}
In the begining of~\cite[Chapter~7.3]{CurryHoward}, one is confronted with a natural question: \textit{``Natural deduction proofs correspond to $\Lam$-terms with types, and Hilbert style proofs correspond to combinators with types. What do sequent calculus proofs correspond to?''}.
This question has its starting point in the well-known Curry-Howard isomorphism, that relates natural deduction proofs with $\Lam$-terms with types, as said above.

Many (naive) alternatives are given in the aforementioned book, but none that can match the process of cut-elimination with normalisation.
In the novel paper of Herbelin~\cite{Herbelin1994}, a multiary version of the $\Lam$-calculus (with explicit substitutions) is introduced, whose typing rules correspond to a fragment of the sequent calculus.
Furthermore, the corrsponding reduction rules for this system behave as cut elimination.

Considering a slightly different multiary version of the $\Lam$-calculus (and excluding explicit substitutions), one gets a system that was studied in detail in CMAT~\cite{JCES2002, JCESLuis}, here named $\LamM$.
The study of the computational meaning behind the sequent calculus is one of the main motivations for considering such systems, as they provide meaningful extensions for the ordinary $\Lam$-calculus.

\section{Objectives}
The theoretical objectives for this dissertation are the study of:
\begin{enumerate}
\item system $\LamM$, reduction rules, typing rules and standard results like subject reduction;
\item system $\LamV$ as subsystem of $\LamM$;
\item the conservativeness of $\LamV$ over $\LamM$;
\item the isomorphism between $\LamV$ and $\Lam$.
\end{enumerate}

We say theoretical objectives because the complete objective is to mechanise in the \textit{Rocq Prover} each of the mentioned items.
The practical (in the sense of the mechanisation task) objectives of this dissertation are first to understand the proof assistant in order to fully develop a mechanisation of the definitions and proofs that were studied using pen-and-paper.
More concretely, we have the objective of understanding how one can use the \textit{Rocq Prover} to define systems that deal with variable binding, to define subsystems, to define typing rules, to prove isomorphisms and so on.

A last objective related to the mechanisation is to implement every definition and proof as close as possible to the pen-and-paper versions, but also to present clean and simple implementations.
Of course this may be challenging and conflictuous, but it is in general the objective behind any mechanisation of mathematics. 

\section{Document structure}

This dissertain is structured in six chapters.

The second chapter serves as an introduction to the ordinary $\Lam$-calculus, also containing a second style of $\Lam$-calculus presentation, without variable names (also called de~Bruijn representation).
This chapter also includes a mechanisation of this system as a way to introduce many proof-assistant concepts used in further chapters.

The third chapter introduces the system $\LamM$ and its canonical subsystem, along with some simple results.
It also includes a last chapter that provides a walkthrough of the mechanised definitions and proofs.

The fourth chapter independenlty introduces a new system called $\LamV$, that is isomorphic to the introduced canonical subsystem of $\LamM$.
This system will help clarify the mechanisation of the result of conservativeness that can be found in this chapter.
The last section also provides a mechanisation overview, similar to the previous chapter.

The fifth chapter is only about the isomorphism between the ordinary $\Lam$-calculus and system $\LamV$.

A last (sixth) chapter with the title of ``Discussion'' contains many directions that were not explored in the development of our work along with some further justification of the choices taken and some possible further work.


%%% Local Variables:
%%% mode: LaTeX
%%% TeX-master: "../dissertation"
%%% End:
