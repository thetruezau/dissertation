\chapter{Introduction}
\label{c:intro}

\section{Motivation}
\subsection{Why mechanise?}
The mechanisation of metatheory has been taken seriously for at least 20 years~\cite{POPLmark}.
By mechanisation we mean a well-founded description of a mathematical object using a proof assistant.
Such proof assistants have attracted the attention of mathematicians because of the reliability they provide for writing computer verified proofs~\cite{FourColourThm}.
There has also been an increasing interest by engineers in the use of such tools for the security guarantees achieved when formally proving properties about computer programmes~\cite{CompCert}.

One could even argue that any work of mechanisation is useful, because it will:
\begin{enumerate}
\item result in a computer verified work,
\item expose the difficulties behind any mathematical formalisation,
\item provide automation for routine and tedious parts,
\item potentially allow some theory to be extended with less cost.
\end{enumerate}

All of the latter are perfectly good motivations for the work done in this dissertation.

Then, the question of \textit{Why metatheory?} arises.
From the reasons above, some are highlighted by the task of mechanising metatheory.
For example, it is often argued that metatheoretical proofs \textit{``are long, contain few essential insights, and have a lot of tedious but error-prone cases''}~\cite{AutosubstSchafer}.

In our case, mechanising theoretical results related to a really specific variation of the $\Lam$-calculus could enable new ways of continuing the work being done in this topic.

\subsection{Why the multiary $\Lam$-calculus?}
In the begining of~\cite[Chapter~7.3]{CurryHoward}, one can read \textit{``Natural deduction proofs correspond to $\Lam$-terms with types, and Hilbert style proofs correspond to combinators with types. What do sequent calculus proofs correspond to?''}.
Many alternatives are given in this chapter, but none that can match the process of cut-elimination with normalisation.

In the novel paper of Herbelin~\cite{Herbelin1994}, a multiary $\Lam$-calculus (with explicit substitutions) is introduced, whose typing rules resemble a sequent calculus style of inference.
Furthermore, the introduced reductions for this system correspond exactly with the process of cut elimination for a fragment of the sequent calculus.

Considering just the multiary version of the $\Lam$-calculus (excluding explicit substitutions), one gets a system that was studied in detail in CMAT~\cite{JCESLuis}, often called $\LamM$.
The study of the computational meaning behind the sequent calculus is one of the main motivations for considering systems such as $\LamM$, as they provide meaningful extensions for the ordinary $\Lam$-calculus.


\section{Objectives}
The initial objectives of this dissertation were clear: to mechanise using the \textit{Rocq Prover} a multiary version of the $\Lam$-calculus and its canonical subsystem.

% (working with inductive structures and so on)
% concrete formalisations
% set theory vs type theory

\section{Document Structure}

%%% Local Variables:
%%% mode: LaTeX
%%% TeX-master: "../dissertation"
%%% End:
