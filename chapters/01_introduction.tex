\chapter{Introduction}
\label{c:intro}

\section{Motivation}
Looking at the title for this dissertation, we can explain what motivated the following dissertation by asking ourselves: ``Why mechanise?'', ``Why metatheory?'' and ``Why the multiary $\Lam$-calculus?''.

Before addressing these questions formally, we could just say that mechanising mathematics is an enjoyable task.
And for what matters, this could be all we say about our motivation.
All I intend to say is that even if our work in mathematics had no application or direct consequences, the fun of doing it would be a good enough motivation.
Mechanising mathematics is like a computer game for a mathematician.

\paragraph{Why mechanise?}
By mechanisation we mean a well-founded description of a mathematical object using a proof assistant.
Such proof assistants have attracted the attention of mathematicians because of the reliability and automation they provide for writing computer verified proofs~\cite{FourColourThm}.
There has also been an increasing interest by engineers in the use of such tools for the security guarantees achieved when formally proving properties about computer programmes~\cite{CompCert}.

One could even argue that any work of mechanisation is useful, because it will:
\begin{enumerate}
\item result in a computer-verified work,
\item expose the difficulties behind any mathematical formalisation,
\item provide automation for routine and tedious parts,
\item potentially allow some theory to be extended with less cost.
\end{enumerate}

Some of the above mentioned items may even be highlighted when the mechanisation refers to metatheory.

\paragraph{Why metatheory?}
It is often argued that metatheoretical proofs \textit{``are long, contain few essential insights, and have a lot of tedious but error-prone cases''}~\cite{AutosubstSchafer}.
This provides fertile ground for computer verification and automation of proofs.
Furthermore, the mechanisation of metatheory has also gained some attention in the past 20 years~\cite{POPLmark, POPLmarkReloaded}.

In our case, mechanising theoretical results related to the multiary $\Lam$-calculus could enable new ways of continuing the work being done in this topic.
Curiously, our work with an unusual version of the $\Lam$-calculus could even suggest some improvements in already mature tools for mechanising metatheory (as is the case with the used \textit{Autosusbt} library for the \textit{Rocq Prover}).

\paragraph{Why the multiary $\Lam$-calculus?}
In the beginning of~\cite[Chapter~7.3]{CurryHoward}, one is confronted with a natural question: \textit{``Natural deduction proofs correspond to $\Lam$-terms with types, and Hilbert style proofs correspond to combinators with types. What do sequent calculus proofs correspond to?''}.
This question has its starting point in the well-known Curry-Howard isomorphism, that relates natural deduction proofs with $\Lam$-terms with types, as said above.

Many (naive) alternatives are given in the aforementioned book, but none that can match the process of cut-elimination with normalisation.
In the novel paper of Herbelin~\cite{Herbelin1994}, a multiary version of the $\Lam$-calculus (with explicit substitutions) is introduced, whose typing rules correspond to a fragment of the sequent calculus and reduction rules behave as cut elimination.

We are interested in a slightly simpler version of the multiary $\Lam$-calculus that excluded explicit substitutions~\cite{JCES2002, JCESLuis}, here named $\LamM$.
Studying the computational meaning behind the sequent calculus is one of the main motivations for considering such systems, as they provide meaningful extensions for the ordinary $\Lam$-calculus.

\section{Objectives and contributions}
The theoretical objectives for this dissertation are the study of:
\begin{enumerate}
\item system $\LamM$, reduction rules, typing rules and standard results like subject reduction;
\item the canonical subsystem of $\LamM$;
\item the conservativeness of the canonical subsystem over $\LamM$;
\item the isomorphism between the canonical subsystem and $\Lam$.
\end{enumerate}

We say theoretical objectives because the complete objective is to mechanise in the \textit{Rocq Prover} each of the mentioned items.
The practical (in the sense of the mechanisation task) objectives of this dissertation are first to understand the proof assistant in order to fully develop a mechanisation of the definitions and proofs that were studied using pen-and-paper.
More concretely, we have the objective of understanding how one can use the \textit{Rocq Prover} to define systems that deal with variable binding, to define subsystems, to define typing rules, to prove isomorphisms and so on.

A last (and challenging) objective related to the mechanisation is to formalise every definition and proof as close as possible to the pen-and-paper versions, assuring clean and simple presentations.

Our contributions of the mechanised metatheory mentioned above can be found online in an open-source \textit{GitHub} repository (\href{https://github.com/thetruezau/LambdaM}{https://github.com/thetruezau/LambdaM}).

\section{Document structure}

This dissertation is structured in six chapters.

% include here a general diagram?

The second chapter serves as an introduction to the ordinary $\Lam$-calculus, also containing a second style of $\Lam$-calculus presentation, without variable names (also called de~Bruijn representation).
This chapter also includes a mechanisation of this system as a way to introduce many proof-assistant concepts used in further chapters.

The third chapter introduces the system $\LamM$ and its canonical subsystem, along with some simple results.
It also includes a last chapter that provides a walk-through of the mechanised definitions and proofs.

The fourth chapter independently introduces a new system called $\LamV$, that is isomorphic to the introduced canonical subsystem of $\LamM$.
This system will help clarify the mechanisation of the result of conservativeness that can be found in this chapter.
The last section also provides a mechanisation overview, similar to the previous chapter.

The fifth chapter is only about the isomorphism between the ordinary $\Lam$-calculus and system $\LamV$.

A last chapter of conclusions lists our contributions, discusses our approaches and related work to then refer some possible future work.

% every chapter refers Rocq scripts that belong to the github repo

%%% Local Variables:
%%% mode: LaTeX
%%% TeX-master: "../dissertation"
%%% End:
