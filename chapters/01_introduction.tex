\chapter{Introduction}
\label{c:intro}

\section{Motivation}
Looking at the title for this dissertation, we can explain what motivated the following dissertation by asking ourselves: ``Why mechanise?'', ``Why metatheory?'' and ``Why the multiary $\Lam$-calculus?''.

Before addressing these questions formally, we could just say that mechanising mathematics is an enjoyable task.
And that could be all we say about our motivation.
Even if our work in mathematics had no application or direct consequences, the fun of mechanising it in a proof assistant would be a good enough motivation.
Mechanising mathematics is like a computer game for a mathematician.

\paragraph{Why mechanise?}
By mechanisation we mean a well-founded description of a mathematical object using a proof assistant.
Such proof assistants have attracted the attention of mathematicians because of the reliability and automation they provide for writing computer verified proofs~\cite{FourColourThm}.
There has also been an increasing interest by engineers in the use of such tools for the security guarantees achieved when formally proving properties about computer programmes~\cite{CompCert}.

One could even argue that any work of mechanisation is useful, because it will:
\begin{enumerate}
\item result in a computer-verified work,
\item expose the difficulties behind any mathematical formalisation,
\item provide automation for routine and tedious parts,
\item potentially allow some theory to be extended with less cost.
\end{enumerate}

Some of the above mentioned items may even be highlighted when the mechanisation refers to metatheory.

\paragraph{Why metatheory?}
It is often argued that metatheoretical proofs \textit{``are long, contain few essential insights, and have a lot of tedious but error-prone cases''}~\cite{AutosubstSchafer}.
This provides fertile ground for computer verification and automation of proofs.
Furthermore, the mechanisation of metatheory has also gained some attention in the past 20 years~\cite{POPLmark, POPLmarkReloaded}.

In our case, mechanising theoretical results related to the multiary $\Lam$-calculus could enable new ways of continuing the work being done in this topic.
Curiously, our work with an unusual version of the $\Lam$-calculus could even suggest some improvements in already mature tools for mechanising metatheory (as is the case with the used \textit{Autosusbt} library for the \textit{Rocq Prover}).

\paragraph{Why the multiary $\Lam$-calculus?}
In the beginning of~\cite[Chapter~7.3]{CurryHoward}, one is confronted with a natural question: \textit{``Natural deduction proofs correspond to $\Lam$-terms with types, and Hilbert style proofs correspond to combinators with types. What do sequent calculus proofs correspond to?''}.
This question has its starting point in the well-known Curry-Howard isomorphism, that relates natural deduction proofs with $\Lam$-terms with types, as said above.

Many (naive) alternatives are given in the aforementioned book, but none that can match the process of cut-elimination with normalisation.
In the novel paper of Herbelin~\cite{Herbelin1994}, a multiary version of the $\Lam$-calculus (with explicit substitutions) called $\overline{\pmb \lambda}$ is introduced, whose typing rules correspond to a fragment of the sequent calculus and reduction rules behave as cut elimination.

We are interested in the study of a slightly different version of $\overline{\pmb \lambda}$ which has no explicit substitutions \cite{JCES2002, JCESLuis}, here named $\LamM$.
Studying the computational meaning behind the sequent calculus is one of the main motivations for considering such systems, as they provide meaningful extensions for the ordinary $\Lam$-calculus.

\section{Objectives and contributions}
The theoretical objectives for this dissertation are the study of:
\begin{enumerate}
\item system $\LamM$, reduction rules, typing rules and standard results like subject reduction;
\item the canonical subsystem of $\LamM$;
\item the conservativeness of the canonical subsystem over $\LamM$;
\item the isomorphism between the canonical subsystem and $\Lam$.
\end{enumerate}

We say theoretical objectives because the complete objective is to mechanise in the \textit{Rocq Prover} each of the mentioned items.
The practical (in the sense of the mechanisation task) objectives of this dissertation are first to understand the proof assistant in order to fully develop a mechanisation of the definitions and proofs that were studied using pen-and-paper.
Concretely, we have the objective of understanding how one can use the \textit{Rocq Prover} to define systems that deal with variable binding, to define subsystems, to define typing rules, to prove isomorphisms and so on.

A last (and challenging) objective related to the mechanisation is to formalise every definition and proof as close as possible to the pen-and-paper versions, assuring clean and simple presentations.

This dissertation presents a mechanised version of the system $\LamM$ and its associated metatheory - including an isomorphism of its subsystem with the simply typed $\Lam$-calculus and the conservativeness results - within the \textit{Rocq Prover}.
We know of no other works formalising this metatheory.
This formalised body of work provides a computer-verified and highly accessible foundation for future study and can be found in an open-source \textit{GitHub} repository \footnote{\href{https://github.com/thetruezau/LambdaM}{https://github.com/thetruezau/LambdaM}}.

\section{Document structure}

This dissertation is organised as follows:

% include here a general diagram?

\paragraph{\cref{c:background}}
serves as an introduction to the ordinary $\Lam$-calculus, also containing a second style of $\Lam$-calculus presentation, without variable names (also called de~Bruijn representation).
This chapter also includes a mechanisation of this system as a way to introduce many proof-assistant concepts used in further chapters.

\paragraph{\cref{c:multiary}} introduces the system $\LamM$ and its canonical subsystem, along with some simple results.
It also includes a last chapter that provides a walk-through of the mechanised definitions and proofs.

\paragraph{\cref{c:canonical}} independently introduces a new system called $\LamV$, that is isomorphic to the introduced canonical subsystem of $\LamM$.
This system will help clarify the mechanisation of the result of conservativeness that can be found in this chapter.
The last section also provides a mechanisation overview, similar to the previous chapter.

\paragraph{\cref{c:isomorphism}} is only about the isomorphism between the ordinary $\Lam$-calculus and system $\LamV$.

\paragraph{\cref{c:conclusions}} lists our contributions, discusses our approaches and related work and then points towards possible future work.

Every \textit{Rocq Prover} module/script referred throughout this document may be found in the previously mentioned \textit{GitHub} repository.

%%% Local Variables:
%%% mode: LaTeX
%%% TeX-master: "../dissertation"
%%% End:
