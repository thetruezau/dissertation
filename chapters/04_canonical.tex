\chapter{Canonical $\Lam$-calculus}
\label{c:canonical}

% ideias:
% este canonical lambda calculus e uma versao autocontida do canonical subsistema (sem referencias ao lambda m)
% no capitulo 5 vamos demonstrar que este sistema 'e isomorfo ao lambda calculus
% este canonical lambda calculus corresponde a um lambda calculus vectorial (dai a sua notacao!)
% estas duas ideais justificam a terminologia

In this chapter we present a system that we give the name of canonical $\Lam$-calculus ($\LamV$).

We call it canonical because it is in fact isomorphic to the canonical subsystem of $\LamM$ introduced in the previous chapter.
Moreover, it is a self-contained representation of the canonical subsystem (one can notice the similarities in the definitions when comparing to system $\LamM$).
We will show this isomorphism in the second section.
In the third section we give proof for the theorem of conservativeness, stating that $\LamM$ is a conservative extension of $\LamV$. 

Furthermore, we call it a canonical $\Lam$-calculus because this system is also isomorphic to the simply typed $\Lam$-calculus.
This is formally argued in \cref{c:isomorphism}.

\section{The system $\LamV$}

\begin{definition}[Syntax of $\LamV$]
  The $\LamV$-terms and $\LamV$-lists are simultaneously defined by the following grammar:
  \begin{align*} 
    t, u \ &::= \ var(x) \ | \ \lambda x.t \ | \ app_{v}(x, u, l) \ | \ app_{\lambda} (x.t, u, l) \\
    l      &::= \ []\  | \ u :: l
  \end{align*}
\end{definition}

\begin{remark}
  In the $\LamV$-terms there exist two different binding constructors: $\lambda x.t$ and $app_{\lambda} (x.t, u, l)$.
  In both constructors, every occurrence of $var(x)$ in the term $t$ is bound and not free.
  This is, system $\LamV$ has a dedicated constructor for the multiary application $(\lambda x.t)(u, l)$ in system $\LamM$.  
\end{remark}

\begin{definition}
  Given $\LamV$-terms $t, u$ and $\LamV$-list $l$, the operation $t@(u, l)$ is defined by the following equations:
  \begin{align*}
    & var(x)@(u, l) = app_v(x, u, l), \\
    & (\lambda x. t)@(u, l) = app_\lambda (x. t, u, l), \\ 
    & app_v(x, u', l')@(u, l) = app_v(x, u', l'+(u::l)) \\
    & app_\lambda (x. t, u', l')@(u, l) = app_\lambda (x. t, u', l'+(u::l)),
  \end{align*}
  where the list append, $l + l'$, is defined simlarly as in $\LamM$.  
\end{definition}

Now follows a strange definition for the substitution operation, as we have to be careful when dealing with a substitution over a constructor $app_v$.

\begin{definition}[Substitution for $\LamV$-terms]
  \label{canonical_substitution}
  The substitution over a $\LamV$-term is mutually defined with the substitution over a $\LamV$-list as follows:
  \begin{align*}
  & var(x)[x := v] = v ; \\
  & var(y)[x := v] = y \text{, with } x \neq y ; \\
  & (\lambda y . t)[x := v] = \lambda y . (t[x := v]) ; \\
  & app_v(x, u, l)[x := v] = v @ (u[x := v], l[x := v]) ; \\
  & app_v(y, u, l)[x := v] = app_v(y, u[x := v], l[x := v]) \text{, with } x \neq y ; \\
  & app_\lambda (y . t, u, l)[x := v] = app_\lambda (y . t[x := v], u[x := v], l[x := v]) ; \\
  & ([])[x := v] = [] ; \\
  & (u::l)[x := v] = u[x := v] :: l[x := v] .
  \end{align*}
\end{definition}


\begin{definition}[Compatible Relation]
  Let $R$ and $R'$ be two binary relations on $\LamV$-terms and $\LamV$-lists respectively.
  We say they are compatible when they satisfy:
  \[
    \begin{prooftree}
      \hypo{ (t, t') \in R }
      \infer1{ (\lambda x . t, \lambda x . t') \in R } 
    \end{prooftree}
    \qquad
    \begin{prooftree}
      \hypo{ (t, t') \in R }
      \infer1{ (app_\lambda (x.t, u, l), app_\lambda (x.t', u, l)) \in R } 
    \end{prooftree}
  \]
  \[
    \begin{prooftree}
      \hypo{ (u, u') \in R }
      \infer1{ (app_\lambda (x.t, u, l), app_\lambda (x.t, u', l)) \in R }
    \end{prooftree}
    \qquad
    \begin{prooftree}
      \hypo{ (l, l') \in R' }
      \infer1{ (app_\lambda (x.t, u, l), app_\lambda (x.t, u, l')) \in R } 
    \end{prooftree}
  \]
  \[
    \begin{prooftree}
      \hypo{ (u, u') \in R }
      \infer1{ (app_v (x, u, l), app_v (x, u', l)) \in R } 
    \end{prooftree}
    \qquad
    \begin{prooftree}
      \hypo{ (l, l') \in R' }
      \infer1{ (app_v (x, u, l), app_v (x, u, l')) \in R } 
    \end{prooftree}
  \]
  \[
    \begin{prooftree}
      \hypo{ (u, u') \in R }
      \infer1{ (u::l, u'::l) \in R' } 
    \end{prooftree}
    \qquad
    \begin{prooftree}
      \hypo{ (l, l') \in R' }
      \infer1{ (u::l, u::l') \in R' } 
    \end{prooftree}
  \]
\end{definition}


\begin{lemma}[Compatibility lemmas]
  \label{compatibility_lemmas}
  Let $R$ and $R'$ be two binary relations on $\LamV$-terms and $\LamV$-lists respectively.
  If $R$ and $R'$ are compatible, then they satisfy:
    \[
    \begin{prooftree}
      \hypo{ (l_1, l_1') \in R' }
      \infer1{ (l_1+l_2, l_1'+l_2) \in R' } 
    \end{prooftree}
    \qquad
    \begin{prooftree}
      \hypo{ (l_2, l_2') \in R' }
      \infer1{ (l_1+l_2, l_1+l_2') \in R' }
    \end{prooftree}
  \]
  \[
    \begin{prooftree}
      \hypo{ (t, t') \in R }
      \infer1{ (t@(u,l), t'@(u,l)) \in R } 
    \end{prooftree}
    \qquad
    \begin{prooftree}
      \hypo{ (u, u') \in R }
      \infer1{ (t@(u,l), t@(u',l)) \in R } 
    \end{prooftree}
    \qquad
    \begin{prooftree}
      \hypo{ (l, l') \in R' }
      \infer1{ (t@(u,l), t@(u,l')) \in R } 
    \end{prooftree}
  \]
\end{lemma}
\begin{proof}
  The proof proceeds easily by induction on lists for the append cases.

  For the compatibility cases of @ operation, proof follows by inspection of the principle argument and application of the append cases. 
\end{proof}

\begin{definition}[Reduction rules for $\LamV$-terms]  
  \begin{align*}
    & app_\lambda (x . t, u, []) \to_{\beta_1} t[x := u]
    \\
    & app_\lambda (x . t, u, v::l) \to_{\beta_2} t[x := u]@(v, l)
  \end{align*}
  % By abuse of notation, we introduce reduction rules with the notation of their compatible closure ($\to_R$).
\end{definition}


\begin{definition}[$\beta$-normal forms]
  \label{beta_nfs_can}
  We inductively define the sets of $\LamV$-terms and $\LamV$-lists in $\beta$-normal form, respectively NT and NL, as follows:
  \[
    \begin{prooftree}
      \infer0{ var(x) \in \text{NT} }
    \end{prooftree}
    \qquad
    \begin{prooftree}
      \hypo{ t \in \text{NT} }
      \infer1{ \lambda x . t \in \text{NT} } 
    \end{prooftree}
    \qquad
    \begin{prooftree}
      \hypo{ u \in \text{NT} } 
      \hypo{ l \in \text{NL} }
      \infer2{ app_v (x, u, l) \in \text{NT} }
    \end{prooftree}
    \qquad
    \begin{prooftree}
      \infer0{ [] \in \text{NL} }
    \end{prooftree}
    \qquad
    \begin{prooftree}
      \hypo{ u \in \text{NT} }
      \hypo{ l \in \text{NL} }
      \infer2{ u::l \in \text{NL} }
    \end{prooftree}.
  \]
\end{definition}

\begin{remark}
  One could simply describe the $\beta$-normal forms of $\LamV$ as the terms and lists with no occurrences of the constructor $app_{\lambda}$.
  This corresponds to idea of cut-elimination from sequent calculus and is one of the motivations for working with such systems. % proof search
  This system also offers an advantage in comparison to the $\Lam$-calculus, where a description of $\beta$-normal forms is more elaborated.
\end{remark}

\begin{claim}
  \label{beta_nfs_can_claim}  
  Given a $\LamV$-term $t$, the following are equivalent:

  (i) $t \in \text{NT}$;

  (ii) $t$ is irreducible by $\to_\beta$.
\end{claim}

We leave a similar claim to~\cref{beta_nfs_claim} that will not be proved here.
However, it will be used in the next chapter to provide an alternative argument for the bijection between $\beta$-normal forms of $\Lam$-terms and $\LamV$-terms.

Sequents in system $\LamV$ may be defined similarly as in system $\LamM$.
Then, we directly introduce the typing rules of this system.


\begin{definition}[Typing Rules for $\LamV$-terms]
  \[
    \begin{prooftree}
      \infer0[Var]{ x:A, \Gamma \vdash var(x):A } 
    \end{prooftree}
    \qquad
    \begin{prooftree}
      \hypo{ x:A, \Gamma \vdash t:B }
      \infer1[Abs]{ \Gamma \vdash \lambda x . t : A \supset B  } 
    \end{prooftree}
  \]
  \[
    \begin{prooftree}
      \hypo{ \Gamma, x:A \supset B \vdash u:A}
      \hypo{ \Gamma, x:A \supset B ; B \vdash l:C }	
      \infer2[VarApp]{ \Gamma, x:A \supset B \vdash app_v (x, u, l) : C } 
    \end{prooftree}
  \]
  \[
    \begin{prooftree}
      \hypo{ \Gamma, x:A \vdash t:B }
      \hypo{ \Gamma \vdash u:A }
      \hypo{ \Gamma ; B \vdash l : C }	
      \infer3[LamApp]{ \Gamma \vdash app_\lambda (x . t, u, l) : C } 
    \end{prooftree}
  \]
  \[
    \begin{prooftree}
      \infer0[Nil]{ \Gamma ; A \vdash []:A } 
    \end{prooftree}
    \qquad
    \begin{prooftree}
      \hypo{ \Gamma \vdash u:A }
      \hypo{ \Gamma ; B \vdash l:C }
      \infer2[Cons]{ \Gamma ; A \supset B \vdash  u::l : C } 
    \end{prooftree}
  \]
\end{definition}

% ---
% 123
% ---

\section{$\LamV$ vs the canonical subsystem of $\LamM$}

In this section we prove an isomorphism between $\LamV$ and the canonical subsystem in $\LamM$. 

\begin{definition}
  Consider the following maps $i$ and $p$:
  \begin{align*}
    i : \LamV \text{-terms} &\to Can \\
    var(x) &\mapsto x \\
    \lambda x . t &\mapsto \lambda x . i(t) \\
    app_v (x, u, l) &\mapsto x(i(u), i'(l)) \\
    app_\lambda (x. t, u, l) &\mapsto (\lambda x . i(t))(i(u), i'(l)),
  \end{align*}
  where $i'$ is simply defined as
  $i'([]) \mapsto []$ and $i'(u::l) = i(u)::i'(l)$;
  \begin{align*}
    p : \LamM \text{-terms} &\to \LamV \text{-terms} \\
    x &\mapsto var(x) \\
    \lambda x . t &\mapsto \lambda x . p(t) \\
    t(u, l) &\mapsto p(t)@(p(u), p'(l)),
  \end{align*}
  where $p'$ is simply defined as
  $p'([]) \mapsto []$ and $p'(u::l) = p(u)::p'(l)$.
\end{definition}

The following diagram summarizes the defined maps.

% https://tikzcd.yichuanshen.de/#N4Igdg9gJgpgziAXAbVABwnAlgFyxMJZARgBoAGAXVJADcBDAGwFcYkQAdDtAWwCNgXRvX5R6AAh4BfEFNLpMufIRRli1Ok1bsuvAQGF6YGXIXY8BIuVLqaDFm0Sdu-QR1owAxuKEi+Ykw0YKABzeCJQADMAJwgeJGsQHAgkMhBhPhhGAAVFCxV0mEicEDstRxAAC1l5EBi4pAAmGmTUsod2LFL0+kycvOV2aKwQypLTOtj4xETWxGbNDqc0boys3PNBp2HR8copIA
\[
  \begin{tikzcd}
    & \pmb{\lambda m} \arrow[d, "h"] \arrow[ld, "p"'] \\
    \pmb{\vec \lambda} \arrow[r, "i"'] & Can                                      
  \end{tikzcd}
\]

% We show some useful lemmas for the following results.
We begin by proving that the diagram shown is commutative.

\begin{lemma}
  \label{i_app_comm}
  Given $\LamV$-terms $t,u$ and $\LamV$-list $l$, 
  \[ i(t@(u,l)) = app(i(t), i(u), i'(l)). \]
\end{lemma}
\begin{proof}
  The proof proceeds easily by inspection of the $\LamV$-term $t$. 
\end{proof}

\begin{theorem}
  \label{comm_i_p_h}
  \begin{align*}
    i \circ p &= h \\
    i' \circ p' &= h'    
  \end{align*}
\end{theorem}
\begin{proof}
  The proof proceeds easily by simultaneous induction on the structure of the $\LamM$-term, using~\cref{i_app_comm} in the application case.
\end{proof}

% We now see that the defined maps establish a bijection between the $\LamV$-terms and the subsyntax of $\LamM$-terms in the set $Can$.

\subsection{Bijection at the level of terms}

\begin{corollary}
  \label{inversion_ip}
  \begin{align*}
    i \circ p|_{Can} &= id_{Can} \\
    i' \circ p'|_{CanList} &= id_{CanList}
  \end{align*}
\end{corollary}
\begin{proof}
  Each inversion is obtained via rewriting with~\cref{h_fixpoints} and then using~\cref{comm_i_p_h}.
\end{proof}

\begin{theorem}
  \label{inversion_pi}
  \begin{align*}
    p \circ i &= id_{\LamV \text{-terms}} \\
    p' \circ i' &= id_{\LamV \text{-terms}}    
  \end{align*}
\end{theorem}
\begin{proof}
  The proof proceeds easily by simultaneous induction on the structure of the $\LamV$-term. 
\end{proof}

% ---
% ---

\subsection{Isomorphism at the level of reduction}

In our subsytem of canonical terms, the substitution is not closed for the substitution operation.
We have the following result that relates the two notions of substitution.

\begin{lemma}
  \label{i_subst_pres}
  For every $\LamV$-terms $t, u$,
  \[ i(t[x := u]) = h(i(t)[x := i(u)]) \]
  and also, for every $\LamV$-term $u$ and $\LamV$-list $l$,
  \[ i'(l[x := u]) = h'(i'(l)[x := i(u)]). \]
\end{lemma}
\begin{proof}
  The proof proceeds easily by simultaneous induction on the structure of the $\LamV$-term $t$.

  For the case where $t = app_v (x, u, l)$, we use~\cref{i_app_comm} to rewrite the term $i(t[x := v]) = i(v@(u,l))$ as $app(i(v), i(u), i'(l))$.
\end{proof}

\begin{lemma}
  \label{p_subst_pres}
  For every $\LamM$-terms $t, u$,
  \[ p(t[x := u]) = p(t)[x := p(u)] \]
  and also, for every $\LamM$-term $u$ and $\LamM$-list $l$,
  \[ p'(l[x := u]) = p'(l)[x := p(u)]. \]
\end{lemma}
\begin{proof}
  The proof proceeds easily by simultaneous induction on the structure of the $\LamM$-term $t$.
\end{proof}

The following technical lemma says that we can derive the compatibilty rules from the system $\LamV$ given the canonoical closure of compatible relation on $\LamM$.

\begin{lemma}
  \label{step_can_is_compatible}
  Let $R$ and $R'$ be two binary relations on $\LamM$-terms and $\LamM$-lists respectively.

  The following binary relations are compatible in $\LamV$:
  \begin{align*}
    I &= \{ (t, t') \ | \ i(t) \ ({\to_R})_c \ i(t'), \text{for every $\LamV$-terms $t, t'$} \} \\
    I' &= \{ (l, l') \ | \ i'(l) \ ({\to_{R'}})_c \ i'(l'), \text{for every $\LamV$-lists $l, l'$} \}
  \end{align*}
\end{lemma}
\begin{proof}
  We provide proof for one of the compatibility cases:
  \[ \begin{prooftree}
      \hypo{ (t, t') \in I }
      \infer1{ (app_\lambda (x.t, u, l), app_\lambda (x.t', u, l)) \in I }
    \end{prooftree} . \]

  From the definition of $I$, $(t, t') \in I \implies i(t) \ ({\to_R})_c \ i(t')$.

  Then, from the definition of the canonical closure relation, we have that there exist $\LamM$-terms $t_1$ and $t_2$ such that $h(t_1) = i(t)$ and $h(t_2) = i(t')$ and $t_1 \to_R t_2$.

  We have:
  \[ \begin{prooftree}
      \hypo{ t_1 \to_R t_2 }
      \infer1[\text{(compatibility of $\to_R$)}]{ \lambda x.t_1 \to_R \lambda x.t_2 }
      \infer1[\text{(compatibility of $\to_R$)}]{ (\lambda x.t_1)(i(u), i'(l)) \to_R (\lambda x.t_2)(i(u), i'(l)) }
      \infer1[\text{(compatibility of $\to_R$)}]{ (\lambda x.t_1)(i(u), i'(l)) \to_R (\lambda x.t_2)(i(u), i'(l)) }
      \infer1[\text{(canonical closure definition)}]{ h((\lambda x.t_1)(i(u), i'(l))) \ ({\to_R})_c \ h((\lambda x.t_2)(i(u), i'(l))) }.
    \end{prooftree} \]

  Computing $h$, we get $(\lambda x.h(t_1))(h(i(u)), h'(i'(l))) \ ({\to_R})_c \ (\lambda x.h(t_2))(h(i(u)), h'(i'(l)))$.

  As $i(u) \in Can$, $h(i(u)) = i(u)$.
  And also, because $i'(l) \in CanList$, we get that $h'(i'(l)) = i'(l)$.
  \begin{align*}
    &(\lambda x.h(t_1))(i(u), i'(l)) = (\lambda x.i(t))(i(u), i'(l)) = i(app_\lambda (x.t, u, l)) \\
    ({\to_R})_c \ &(\lambda x.h(t_2))(i(u), i'(l)) = (\lambda x.i(t'))(i(u), i'(l)) = i(app_\lambda (x.t', u, l))
  \end{align*}

  Therefore, by definition of $I$, we get that $(app_\lambda (x.t, u, l), app_\lambda (x.t', u, l)) \in I$.
\end{proof}

\begin{theorem}
  \label{i_step_pres}
  For every $\LamV$-terms $t, t'$,
  \[ t \to_{\beta} t' \implies i(t) \ ({\to_{\beta}})_c \ i(t') \]
  and also, for every $\LamV$-lists $l, l'$,
  \[ l \to_{\beta} l' \implies i'(l) \ ({\to_{\beta}})_c \ i(l'). \]
\end{theorem}
\begin{proof}
  The proof proceeds by simultaneous induction on the step relation of $\LamV$-terms.

  \cref{i_subst_pres} deals with substitution preservation in the $\beta$-reduction cases.

  \cref{step_can_is_compatible} deals with all the compatibility cases.
\end{proof}

\begin{theorem}
  \label{p_step_pres}
  For every $t, t' \in Can$,
  \[ t \ ({\to_{\beta}})_c \ t' \implies p(t) \to_{\beta} p(t') \]
  and also, for every $l, l' \in CanList$,
  \[ l \ ({\to_{\beta}})_c \ l' \implies p'(l) \to_{\beta} p(l'). \]
\end{theorem}
\begin{proof}
  The proof proceeds by simultaneous induction on the step relation of canonical terms.
  
  \cref{compatibility_lemmas} may be useful for compatibility steps.

  \cref{p_subst_pres} deals with substitution preservation in the $\beta$-reduction cases.  
\end{proof}

% ---
% 123
% ---

\subsection{Isomorphism at the level of typed terms}

\begin{lemma}[Append admissibility]
  \label{append_admissibility}
  The following rule is admissible in $\LamV$:
  \[ \begin{prooftree}
      \hypo{ \Gamma ; A \vdash l : B }
      \hypo{ \Gamma ; B \vdash l' : C }
      \infer2{ \Gamma ; A \vdash l+l' : C } 
    \end{prooftree} . \]
\end{lemma}
\begin{proof}
  The proof proceeds easily by induction on the list $l$.
\end{proof}


\begin{lemma}[@ admissibility]
  \label{app_admissibility}
  The following rule is admissible in $\LamV$:
  \[ \begin{prooftree}
      \hypo{ \Gamma \vdash t : A \supset B }
      \hypo{ \Gamma \vdash u : A }
      \hypo{ \Gamma ; B \vdash l : C }      
      \infer3{ \Gamma \vdash t@(u,l) : C } 
    \end{prooftree} . \]
\end{lemma}
\begin{proof}
  The proof proceeds easily by inspection of $t$, using \cref{append_admissibility} when $t$ is an application. 
\end{proof}


\begin{theorem}[$i$ admissibility]
  \label{i_admissibility}
    For every $\LamV$-term $t$ and $\LamV$-list $l$, the following rules are admissible:
  \[ \begin{prooftree}
      \hypo{ \Gamma \vdash t : A }
      \infer1{ \Gamma \vdash_c i(t) : A } 
    \end{prooftree}
    \qquad \qquad \qquad
    \begin{prooftree}
      \hypo{ \Gamma ; A \vdash l : B }
      \infer1{ \Gamma ; A \vdash_c i'(l) : B } 
    \end{prooftree}. \]
\end{theorem}
\begin{proof}
  The proof proceeds easily by simultaneous induction on the typing rules of $\LamV$.
\end{proof}


\begin{theorem}[$p$ admissibility]
  \label{p_admissibility}
  For every $t \in Can$ and $l \in CanList$, the following rules are admissible:
  \[ \begin{prooftree}
      \hypo{ \Gamma \vdash_c t : A }
      \infer1{ \Gamma \vdash p(t) : A } 
    \end{prooftree}
    \qquad \qquad \qquad
    \begin{prooftree}
      \hypo{ \Gamma ; A \vdash_c l : B }
      \infer1{ \Gamma ; A \vdash p'(l) : B } 
    \end{prooftree}. \]
\end{theorem}
\begin{proof}
  From~\cref{h_fixpoints} we have that $h(t) = t$ and $h'(l) = l$.

  Then, inverting~\cref{canonical_typing}, we have (in $\LamM$):
  \[ \Gamma \vdash t : A
    \qquad \qquad \qquad
    \Gamma ; A \vdash l : B . \]
  
  Therefore, the proof proceeds easily by simultaneous induction on the typing rules of $\LamM$.

  \cref{app_admissibility} is crucial for the application case.
\end{proof}

Our argument for the isomorphism between the canonical subsystem in $\LamM$ and $\LamV$ ends here.

From now on, we will use the self contained representation, system $\LamV$, to talk about canonical terms.

% ---
% 123
% ---

\section{Conservativeness}

The result of conservativeness establishes the connection between reduction in $\LamV$ and in $\LamM$.

\begin{theorem}[Conservativeness]
  \label{conservativeness}
  For every $\LamV$-terms $t$ and $t'$, we have:
  \[
    t \twoheadrightarrow_\beta t' \iff
    i(t) \twoheadrightarrow_{\beta h} i(t')
  \]
\end{theorem}
\begin{proof}
  $\boxed \implies~$
  Let $t$ and $t'$ be $\LamV$-terms.

  For this implication it suffices to mimic $\beta$ steps of the system $\LamV$ in the system $\LamM$.

  Case $t \to_{\beta_1} t'$:
  % https://q.uiver.app/#q=WzAsMTIsWzAsMiwiYXBwX3tcXGxhbWJkYX0oeC50LCB1LCBbXSkiXSxbMCw0LCJ0W3g6PXVdIl0sWzEsMSwiKFxcbGFtYmRhIHguIGkodCkpKGkodSksIGknKGwpKSJdLFsxLDUsImkodFt4Oj11XSkiXSxbMSwyLCJpKHQpW3ggOj0gaSh1KV0iXSxbMSw0LCJoKGkodClbeDo9aSh1KV0pIl0sWzAsMCwiKFxcdGV4dHtpbn0gXFwgXFx2ZWMgXFxsYW1iZGEpIl0sWzEsMCwiKFxcdGV4dHtpbn0gXFwgXFxsYW1iZGEgbSkiXSxbMCwzXSxbMiwzXSxbMiw0XSxbMiw1XSxbMCwyLCJpIiwxXSxbMSwzLCJpIiwxXSxbMiw0LCJfe1xcYmV0YV8xfSIsMl0sWzUsMywiIiwyLHsibGV2ZWwiOjIsInN0eWxlIjp7ImhlYWQiOnsibmFtZSI6Im5vbmUifX19XSxbNCw1LCJfaCIsMix7InN0eWxlIjp7ImhlYWQiOnsibmFtZSI6ImVwaSJ9fX1dLFswLDEsIl97XFxiZXRhXzF9IiwyXSxbOSwxMCwibGVtbWEiLDEseyJzdHlsZSI6eyJib2R5Ijp7Im5hbWUiOiJub25lIn0sImhlYWQiOnsibmFtZSI6Im5vbmUifX19XSxbMTAsMTEsImxlbW1hIiwxLHsic3R5bGUiOnsiYm9keSI6eyJuYW1lIjoibm9uZSJ9LCJoZWFkIjp7Im5hbWUiOiJub25lIn19fV1d
  \[\begin{tikzcd}
	{(\text{in} \ \LamV)} & {(\text{in} \ \LamM)} \\
	& {(\lambda x. i(t))(i(u), i'(l))} \\
	{app_{\lambda}(x.t, u, [])} & {i(t)[x := i(u)]} \\
	{} && {} \\
	{t[x:=u]} & {h(i(t)[x:=i(u)])} & {} \\
	& {i(t[x:=u])} & {}
	\arrow["{_{\beta_1}}"', from=2-2, to=3-2]
	\arrow["i"{description}, maps to, from=3-1, to=2-2]
	\arrow["{_{\beta_1}}"', from=3-1, to=5-1]
	\arrow["{_h}"', two heads, from=3-2, to=5-2]
	\arrow["\text{\cref{h_is_multistep}}"{description}, draw=none, from=4-3, to=5-3]
	\arrow["i"{description}, maps to, from=5-1, to=6-2]
	\arrow[equals, from=5-2, to=6-2]
	\arrow["\text{\cref{i_subst_pres}}"{description}, draw=none, from=5-3, to=6-3]
      \end{tikzcd}\]
  
  Case $t \to_{\beta_2} t'$:
  % https://q.uiver.app/#q=WzAsMTUsWzEsMiwiKGkodClbeCA6PSBpKHUpXSkoaSh2KSwgaScobCkpIl0sWzAsMCwiXFxzbWFsbHsoXFx0ZXh0e2lufSBcXCBcXHZlYyBcXGxhbWJkYSl9Il0sWzEsMCwiXFxzbWFsbHsoXFx0ZXh0e2lufSBcXCBcXGxhbWJkYSBtKX0iXSxbMSwxLCIoXFxsYW1iZGEgeC4gaSh0KSkoaSh1KSwgaSh2KTo6aScobCkpIl0sWzEsMywiaChpKHQpW3ggOj0gaSh1KV0pKGkodiksIGknKGwpKSJdLFsxLDYsImkodFt4Oj11XUAodixsKSkiXSxbMSw1LCJhcHAoaSh0W3g6PXVdKSwgaSh2KSwgaScobCkpIl0sWzEsNCwiKGkodFt4IDo9IHVdKShpKHYpLCBpJyhsKSkiXSxbMCwyLCJhcHBfe1xcbGFtYmRhfSh4LnQsIHUsIHY6OmwpIl0sWzAsNSwidFt4Oj11XUAodiwgbCkiXSxbMiw1XSxbMiw0XSxbMiwzXSxbMiwyXSxbMiw2XSxbMywwLCJfe1xcYmV0YV8yfSJdLFswLDQsIl9oIiwwLHsic3R5bGUiOnsiaGVhZCI6eyJuYW1lIjoiZXBpIn19fV0sWzQsNywiIiwwLHsic2hvcnRlbiI6eyJzb3VyY2UiOjMwLCJ0YXJnZXQiOjMwfSwibGV2ZWwiOjIsInN0eWxlIjp7ImhlYWQiOnsibmFtZSI6Im5vbmUifX19XSxbNiw1LCIiLDEseyJzaG9ydGVuIjp7InNvdXJjZSI6MzAsInRhcmdldCI6MzB9LCJsZXZlbCI6Miwic3R5bGUiOnsiaGVhZCI6eyJuYW1lIjoibm9uZSJ9fX1dLFs3LDYsIl9oIiwwLHsic3R5bGUiOnsiaGVhZCI6eyJuYW1lIjoiZXBpIn19fV0sWzgsMywiaSIsMV0sWzksNSwiaSIsMV0sWzgsOSwiX3tcXGJldGFfMn0iXSxbMTEsMTAsImxlbW1hIiwxLHsic3R5bGUiOnsiYm9keSI6eyJuYW1lIjoibm9uZSJ9LCJoZWFkIjp7Im5hbWUiOiJub25lIn19fV0sWzEyLDExLCJsZW1tYSBcXCBpc3Vic3RwcmVzIiwxLHsic3R5bGUiOnsiYm9keSI6eyJuYW1lIjoibm9uZSJ9LCJoZWFkIjp7Im5hbWUiOiJub25lIn19fV0sWzEzLDEyLCJsZW1tYSIsMSx7InN0eWxlIjp7ImJvZHkiOnsibmFtZSI6Im5vbmUifSwiaGVhZCI6eyJuYW1lIjoibm9uZSJ9fX1dLFsxMCwxNCwibGVtbWEgXFwgaWFwcGNvbW0iLDEseyJzdHlsZSI6eyJib2R5Ijp7Im5hbWUiOiJub25lIn0sImhlYWQiOnsibmFtZSI6Im5vbmUifX19XV0=
  \[\begin{tikzcd}
	{\small{(\text{in} \ \LamV)}} & {\small{(\text{in} \ \LamM)}} \\
	& {(\lambda x. i(t))(i(u), i(v)::i'(l))} \\
	{app_{\lambda}(x.t, u, v::l)} & {(i(t)[x := i(u)])(i(v), i'(l))} & {} \\
	& {h(i(t)[x := i(u)])(i(v), i'(l))} & {} \\
	& {(i(t[x := u]))(i(v), i'(l))} & {} \\
	{t[x:=u]@(v, l)} & {app(i(t[x:=u]), i(v), i'(l))} & {} \\
	& {i(t[x:=u]@(v,l))} & {}
	\arrow["{_{\beta_2}}", from=2-2, to=3-2]
	\arrow["i"{description}, maps to, from=3-1, to=2-2]
	\arrow["{_{\beta_2}}", from=3-1, to=6-1]
	\arrow["{_h}", two heads, from=3-2, to=4-2]
	\arrow["\text{\cref{h_is_multistep}}"{description}, draw=none, from=3-3, to=4-3]
	\arrow[equals, from=4-2, to=5-2]
	\arrow["\text{\cref{i_subst_pres}}"{description}, draw=none, from=4-3, to=5-3]
	\arrow["{_h}", two heads, from=5-2, to=6-2]
	\arrow["\text{\cref{app_is_multistep}}"{description}, draw=none, from=5-3, to=6-3]
	\arrow["i"{description}, maps to, from=6-1, to=7-2]
	\arrow[equals, from=6-2, to=7-2]
	\arrow["\text{\cref{i_app_comm}}"{description}, draw=none, from=6-3, to=7-3]
      \end{tikzcd}\]
  
    $\boxed \Longleftarrow$
    Let $t$ and $t'$ be $\LamM$-terms.    
    
  For this implication, we first show how a reduction $t \to_{\beta h} t'$ in $\LamM$ is directly translated to a reduction $p(t) \to_{\beta} p(t')$ in $\LamV$. 
  
  Case $t \to_{\beta_1} t'$:
  % https://q.uiver.app/#q=WzAsMTAsWzAsMiwiKFxcbGFtYmRhIHgudCkodSwgW10pIl0sWzAsMywidFt4Oj11XSJdLFsxLDIsImFwcF97XFxsYW1iZGF9ICh4LnAodCksIHAodSksIFtdKSJdLFsxLDMsInAodClbeDo9cCh1KV0iXSxbMSw0LCJwKHRbeDo9dV0pIl0sWzEsMSwiKFxcbGFtYmRhIHgucCh0KSlAKHAodSksIFtdKSJdLFswLDAsIihcXHRleHR7aW59IFxcIFxcTGFtTSkiXSxbMSwwLCIoXFx0ZXh0e2lufSBcXCBcXExhbVYpIl0sWzIsM10sWzIsNF0sWzAsMSwiX3tcXGJldGFfMX0iXSxbMiwzLCJfe1xcYmV0YV8xfSJdLFszLDQsIiIsMix7ImxldmVsIjoyLCJzdHlsZSI6eyJoZWFkIjp7Im5hbWUiOiJub25lIn19fV0sWzAsNSwicCIsMV0sWzUsMiwiIiwxLHsibGV2ZWwiOjIsInN0eWxlIjp7ImhlYWQiOnsibmFtZSI6Im5vbmUifX19XSxbMSw0LCJwIiwxXSxbOCw5LCJcXHRleHR7XFxjcmVme3Bfc3Vic3RfcHJlc319IiwxLHsic3R5bGUiOnsiYm9keSI6eyJuYW1lIjoibm9uZSJ9LCJoZWFkIjp7Im5hbWUiOiJub25lIn19fV1d
  \[\begin{tikzcd}
	{(\text{in} \ \LamM)} & {(\text{in} \ \LamV)} \\
	& {(\lambda x.p(t))@(p(u), [])} \\
	{(\lambda x.t)(u, [])} & {app_{\lambda} (x.p(t), p(u), [])} \\
	{t[x:=u]} & {p(t)[x:=p(u)]} & {} \\
	& {p(t[x:=u])} & {}
	\arrow[equals, from=2-2, to=3-2]
	\arrow["p"{description}, maps to, from=3-1, to=2-2]
	\arrow["{_{\beta_1}}", from=3-1, to=4-1]
	\arrow["{_{\beta_1}}", from=3-2, to=4-2]
	\arrow["p"{description}, maps to, from=4-1, to=5-2]
	\arrow[equals, from=4-2, to=5-2]
	\arrow["{\text{\cref{p_subst_pres}}}"{description}, draw=none, from=4-3, to=5-3]
      \end{tikzcd}\]

    Case $t \to_{\beta_2} t'$:
    % https://q.uiver.app/#q=WzAsMTAsWzAsMiwiKFxcbGFtYmRhIHgudCkodSwgdjo6bCkiXSxbMCwzLCJ0W3g6PXVdKHYsIGwpIl0sWzEsMiwiYXBwX3tcXGxhbWJkYX0gKHgucCh0KSwgcCh1KSwgcCcobCkpIl0sWzEsMywicCh0KVt4Oj1wKHUpXUAocCh2KSwgcCcobCkpIl0sWzEsNCwicCh0W3g6PXVdKUAocCh2KSwgcCcobCkpIl0sWzEsMSwiKFxcbGFtYmRhIHgucCh0KSlAKHAodSksIHAnKGwpKSJdLFswLDAsIihcXHRleHR7aW59IFxcIFxcTGFtTSkiXSxbMSwwLCIoXFx0ZXh0e2lufSBcXCBcXExhbVYpIl0sWzIsM10sWzIsNF0sWzAsMSwiX3tcXGJldGFfMn0iXSxbMiwzLCJfe1xcYmV0YV8yfSJdLFszLDQsIiIsMix7ImxldmVsIjoyLCJzdHlsZSI6eyJoZWFkIjp7Im5hbWUiOiJub25lIn19fV0sWzAsNSwicCIsMV0sWzUsMiwiIiwxLHsibGV2ZWwiOjIsInN0eWxlIjp7ImhlYWQiOnsibmFtZSI6Im5vbmUifX19XSxbMSw0LCJwIiwxXSxbOCw5LCJcXHRleHR7XFxjcmVme2NvbXBhdGliaWxpdHlfbGVtbWFzfSBhbmQgXFxjcmVme3Bfc3Vic3RfcHJlc319IiwxLHsic3R5bGUiOnsiYm9keSI6eyJuYW1lIjoibm9uZSJ9LCJoZWFkIjp7Im5hbWUiOiJub25lIn19fV1d
    \[\begin{tikzcd}
	{(\text{in} \ \LamM)} & {(\text{in} \ \LamV)} \\
	& {(\lambda x.p(t))@(p(u), p'(l))} \\
	{(\lambda x.t)(u, v::l)} & {app_{\lambda} (x.p(t), p(u), p'(l))} \\
	{t[x:=u](v, l)} & {p(t)[x:=p(u)]@(p(v), p'(l))} & {} \\
	& {p(t[x:=u])@(p(v), p'(l))} & {}
	\arrow[equals, from=2-2, to=3-2]
	\arrow["p"{description}, maps to, from=3-1, to=2-2]
	\arrow["{_{\beta_2}}", from=3-1, to=4-1]
	\arrow["{_{\beta_2}}", from=3-2, to=4-2]
	\arrow["p"{description}, maps to, from=4-1, to=5-2]
	\arrow[equals, from=4-2, to=5-2]
	\arrow["{\text{\cref{compatibility_lemmas} and \cref{p_subst_pres}}}"{description}, draw=none, from=4-3, to=5-3]
      \end{tikzcd}\]

    Case $t \to_h t'$:
    % https://q.uiver.app/#q=WzAsOCxbMCwxLCJ0KHUsIGwpKHUnLCBsJykiXSxbMCwyLCJ0KHUsIGwrKHUnOjpsJykpIl0sWzEsMSwiKHAodClAKHAodSksIHAnKGwpKSlAKHAodScpLCBwJyhsJykpIl0sWzEsMiwicCh0KUAocCh1KSwgcCcobCsodSc6OmwnKSkpIl0sWzEsMCwiKFxcdGV4dHtpbn0gXFwgXFxMYW1WKSJdLFswLDAsIihcXHRleHR7aW59IFxcIFxcTGFtTSkiXSxbMiwxXSxbMiwyXSxbMCwxLCJfaCJdLFswLDIsInAiLDFdLFsxLDMsInAiLDFdLFsyLDMsIiIsMSx7ImxldmVsIjoyLCJzdHlsZSI6eyJoZWFkIjp7Im5hbWUiOiJub25lIn19fV0sWzYsNywiXFx0ZXh0e1xcY3JlZnt1bmtub3duX2xlbW1hfX0iLDEseyJzdHlsZSI6eyJib2R5Ijp7Im5hbWUiOiJub25lIn0sImhlYWQiOnsibmFtZSI6Im5vbmUifX19XV0=
    \[\begin{tikzcd}
	{(\text{in} \ \LamM)} & {(\text{in} \ \LamV)} \\
	{t(u, l)(u', l')} & {(p(t)@(p(u), p'(l)))@(p(u'), p'(l'))} & {} \\
	{t(u, l+(u'::l'))} & {p(t)@(p(u), p'(l+(u'::l')))} & {}
	\arrow["p"{description}, maps to, from=2-1, to=2-2]
	\arrow["{_h}", from=2-1, to=3-1]
	\arrow[equals, from=2-2, to=3-2]
	\arrow["{\text{Simple induction on~$p(t)$}}"{description}, draw=none, from=2-3, to=3-3]
	\arrow["p"{description}, maps to, from=3-1, to=3-2]
      \end{tikzcd}\]

    From these cases, we proved that:
    \begin{align*}
      & t \twoheadrightarrow_{\beta h} t' \implies p(t) \twoheadrightarrow_{\beta} p(t'), \ \text{for every $\LamM$-terms $t, t'$} \\
      \text{(which implies)} \quad & i(t) \twoheadrightarrow_{\beta h} i(t') \implies p(i(t)) \twoheadrightarrow_{\beta} p(i(t')), \ \text{for every $\LamV$-terms $t, t'$} \\
      \text{(simplifying)} \quad & i(t) \twoheadrightarrow_{\beta h} i(t') \implies t \twoheadrightarrow_{\beta} t', \ \text{for every $\LamV$-terms $t, t'$}
    \end{align*}
  \end{proof}

As a corollary of conservativeness, we can derive subject reduction for $\LamV$ from $\LamM$.

\begin{corollary}[Subject Reduction for $\LamV$]
  Given $\LamV$-terms $t$ and $t'$, the follwing holds:
  \[
    \Gamma \vdash t : A \ \land \ t \to_{\beta} t' \implies \Gamma \vdash t' : A.
  \]
\end{corollary}
\begin{proof}
  The proof is shown in a derivation-like style.
  We use dashed lines for derivations that do not follow from typing rules or rules proven admissible.
  \[ \begin{prooftree}      
      \hypo{ \Gamma \vdash t:A }
      \infer[left label=\text{\cref{i_admissibility}}]1{ \Gamma \vdash_c i(t):A }
      \infer[dashed,left label=\text{Inversion of \cref{canonical_typing}}]1{ \Gamma \vdash t_0:A }
      \hypo{ t_0 \twoheadrightarrow_{h} h(t_0) }
      \infer[dashed,left label=\text{\cref{type_preservation} with $\twoheadrightarrow$}]2{ \Gamma \vdash i(t):A }
      \hypo{ t \to_{\beta} t' }
      \infer[dashed]1[\text{\cref{conservativeness}}]{ i(t) \twoheadrightarrow_{\beta h} i(t') }
      \infer[dashed]2[\text{\cref{type_preservation} with $\twoheadrightarrow$}]{ \Gamma \vdash i(t'):A }
      \infer1[\text{\cref{canonical_typing}}]{ \Gamma \vdash_c h(i(t')):A }
      \infer[dashed]1[\text{\cref{h_fixpoints}}]{ \Gamma \vdash_c i(t'):A }
      \infer1[\text{\cref{p_admissibility}}]{ \Gamma \vdash p(i(t')):A }
      \infer[dashed]1[\text{\cref{inversion_pi}}]{ \Gamma \vdash t':A }
    \end{prooftree} \]
\end{proof}

\section{Mechanisation in \textit{Rocq}}

The mechanisations for the system $\LamV$ follow the same style as the ones shown for the system $\LamM$ using the \textit{Autosubst} library, except for the nonstandard substitution operation (that we cover in more detail by the end of the chapter).

\subsection{\lst$Canonical.v$}

Most definitions for the canonical self-contained subsystem follow exactly from the definitions for the system $\LamM$ with the corresponding adaptations.
\begin{lstlisting}[language=Coq]
(* syntax *)
Inductive term: Type :=
| Vari (x: var)
| Lamb (t: {bind term})
| VariApp (x: var) (u: term) (l: list term)
| LambApp (t: {bind term}) (u: term) (l: list term).
...
(* reduction relations *)
Inductive beta1: relation term :=
| Step_Beta1 (t: {bind term}) (t' u: term) :
  t' = t.[u .: ids] -> beta1 (LambApp t u []) t'.

Inductive beta2: relation term :=
| Step_Beta2 (t: {bind term}) (t' u v: term) l :
  t' = t.[u .: ids]@(v,l) -> beta2 (LambApp t u (v::l)) t'.

Definition step := comp (union _ beta1 beta2).
Definition step' := comp' (union _ beta1 beta2).

Definition multistep := clos_refl_trans_1n _ step.
Definition multistep' := clos_refl_trans_1n _ step'.
...
(* typing rules *)
Inductive sequent (Gamma: var->type) : term -> type -> Prop := 
| varAxiom (x: var) (A: type) :
  Gamma x = A -> sequent Gamma (Vari x) A
| Right (t: term) (A B: type) :
  sequent (A .: Gamma) t B -> sequent Gamma (Lamb t) (Arr A B)
| Left (x: var) (u: term) (l: list term) (A B C: type) :
  Gamma x = (Arr A B) -> sequent Gamma u A -> list_sequent Gamma B l C ->
  sequent Gamma (VariApp x u l) C
| KeyCut (t: {bind term}) (u: term) (l: list term) (A B C: type) :
  sequent (A .: Gamma) t B -> sequent Gamma u A -> list_sequent Gamma B l C ->
  sequent Gamma (LambApp t u l) C
with list_sequent (Gamma:var->type) : type -> (list term) -> type -> Prop :=
| nilAxiom (C: type) : list_sequent Gamma C [] C
| Lft (u: term) (l: list term) (A B C:type) :
  sequent Gamma u A -> list_sequent Gamma B l C ->
  list_sequent Gamma (Arr A B) (u :: l) C.
\end{lstlisting}

The mechanised step relations work as shown for the system $\LamM$ using a \lst$comp$ meta-relation for compatibility closure.
In the next subsection we describe in more detail the aproach used to define the substitution operation for this system.

This module also contains proofs for every compatibility lemma (recall~\cref{compatibility_lemmas}).
\begin{lstlisting}[language=Coq]
Section CompatibilityLemmas.
  Lemma step_comp_append1 :
    forall l1 l1', step' l1 l1' -> forall l2, step' (l1 ++ l2) (l1' ++ l2).
  Proof.
    intros l1 l1' H.
    induction H ; intros.
    - repeat rewrite<- app_comm_cons. 
      now constructor. 
    - repeat rewrite<- app_comm_cons.
      constructor. now apply IHcomp'.
  Qed.    
  ...
  Lemma step_comp_app2 :
    forall v u u' l, step u u' -> step v@(u,l) v@(u',l).
  ...
End CompatibilityLemmas.  
\end{lstlisting}

\subsection{\lst$CanonicalIsomorphism.v$}

This module contains every proof related to the isomorphism of the canonical subsystem in $\LamM$ and the system $\LamV$.

Let us see the statement of~\cref{step_can_is_compatible}:
\begin{lstlisting}[language=Coq]
Lemma step_can_is_compatible :
  Canonical.is_compatible
    (fun t t' => step_can (i t) (i t'))
    (fun l l' => step_can' (map i l) (map i l')).
Proof.
  split ; intros ; asimpl ; inversion H.
  ...
\end{lstlisting}

We prove every compatibility step by inverting first the definition of \lst$step_can$.
Despite being a bureocratic result, it helps simplifying further proofs and reveals some benefits of the aproach taken, by formalising the concept of \lst$is_compatible$, for example.

\begin{lstlisting}[language=Coq]
Theorem i_step_pres :
  (forall (t t': Canonical.term),
      Canonical.step t t' -> step_can (i t) (i t'))
  /\
  (forall (l l': list Canonical.term),
      Canonical.step' l l' ->
      step_can' (map i l) (map i l')).
Proof.
  pose step_can_is_compatible as Hic.
  destruct Hic.
  apply Canonical.mut_comp_ind ; intros ; auto.
  ...
\end{lstlisting}

As already mentioned in the last chapter, the mechanised version of~\cref{i_step_pres} makes use of the automation provided by the \lst$auto$ tactic by strategically adding relevant lemmas to the proof context.

\subsection{\lst$Conservativeness.v$}

This module is only about the proof for the conservativeness theorem.
The mechanised theorem follows exactly the proof given diagramatically in~\cref{conservativeness}, divided into two parts, \lst$conservativeness1$ and \lst$conservativeness2$, for each implication side.

\begin{lstlisting}[language=Coq]
Theorem conservativeness :
  forall t t', Canonical.multistep t t' <-> LambdaM.multistep (i t) (i t').
Proof.
  split.
  - intro H.
    induction H as [| t1 t2 t3].
    + constructor.
    + apply multistep_trans with (i t2) ; try easy.
      * now apply conservativeness1. 
  - intro H.
    rewrite<- (proj1 inversion2) with t.
    rewrite<- (proj1 inversion2) with t'.
    induction H as [| t1 t2 t3].
    + constructor.
    + apply multistep_trans with (p t2) ; try easy.
      * now apply conservativeness2. 
Qed.
\end{lstlisting}
  
\subsection{A closer look at the mechanisation}

\subsubsection{\textit{Autosubst} and a nonstandard substitution operation}

One of the most peculiar definitions in system $\LamV$ is the substitution operation.
As referred upon~\cref{canonical_substitution}, we have a strange behaved substitution for the constructor $app_v$.
In practice, on a substitution $app_v (x, u, l)[x:=v]$, there occurs an inspection of the term $v$ that dictates the result of the substitution operation.

As we were working with a library that tries to automate definitions for substitution operations, we tested it in this case.
But as expected, the \lst$derive$ tactic failed to give us the desired operation.
\begin{lstlisting}[language=Coq]
Subst_term =
(fix dummy (sigma : var -> term) (s : term) {struct s} : term :=
match s as t return (annot term t) with
   | Vari x => (fun x0: var => sigma x0) x
   | Lamb t => (fun t0: {bind term} => Lamb t0.[up sigma]) t
   | VariApp x u l => (fun (x0: var) (_: term) (_: list term) => sigma x0) x u l
   | LambApp t u l =>
       (fun (t0: {bind term}) (s0: term) (l0: list term) =>
        LambApp t0.[up sigma] s0.[sigma] l0..[sigma]) t u l
   end)
\end{lstlisting}

Therefore, we gave the proof assistant our dedicated definition (directly as a proof object, as seen below).
\begin{lstlisting}[language=Coq]
Definition app (t u: term) (l: list term): term :=
  match t with
  | Vari x => VariApp x u l
  | Lamb t' => LambApp t' u l
  | VariApp x u' l' => VariApp x u' (l' ++ u::l)
  | LambApp t' u' l' => LambApp t' u' (l' ++ u::l)
  end.
Notation "t '@(' u ',' l ')'" := (app t u l) (at level 9).
...
Instance Ids_term : Ids term. derive. Defined.
Instance Rename_term : Rename term. derive. Defined.
Instance Subst_term : Subst term. 
Proof.
  unfold Subst. fix inst 2. change _ with (Subst term) in inst.
  intros sigma s. change (annot term s). destruct s.
  - exact (sigma x).
  - exact (Lamb (subst (up sigma) t)).
  - exact ((sigma x)@(subst σ s, mmap (subst sigma) l)).
  - exact (LambApp (subst (up sigma) t) (subst sigma s) (mmap (subst sigma) l)).
Defined.
\end{lstlisting}

The downside to this was the need to manually prove every substitution lemma required by the \textit{Autosubst} instance \lst$SubstLemmas$.
This was crucial to enjoy the automation provided from the library for the mechanised inductive type of the $\LamV$-terms.

%%% Local Variables:
%%% mode: LaTeX
%%% TeX-master: "../dissertation"
%%% End:
