\chapter{The isomorphism $\Lam \cong \LamV$}
\label{c:isomorphism}

In~\cref{c:background}, the simply typed $\Lam$-calculus was introduced.
Now, we show an isomorphism between the system $\LamV$ introduced in the previous chapter and the simply typed $\Lam$-calculus.
This isomorphism will come at the level of terms, reduction, $\beta$-normal forms and typing rules.

This isomorphism is of great interest as $\LamV$ typing rules resemble a sequent calculus style.
In this sense, the isomorphism studided in this chapter establishes a correspondence between natural deduction system (the simply typed $\Lam$-calculus) and a fragment of sequent calculus (the system $\LamV$).
The chapter is inspired in the works~\cite{LuisDychkoff} and~\cite[Chapter~4]{JCES2002} and is summarized by the following diagram:

% https://q.uiver.app/#q=WzAsNCxbMiwwLCJcXGJveGVke1xcdmVjIFxcbGFtYmRhfSJdLFsyLDIsIlxcYm94ZWR7XFxzbWFsbHtcXHZlYyBcXGJldGEtbmZzfX0iXSxbMCwwLCJcXGJveGVkXFxsYW1iZGEiXSxbMCwyLCJcXGJveGVke1xcc21hbGx7XFxiZXRhLW5mc319Il0sWzAsMSwiXFxkb3duYXJyb3dfe1xcdmVjIFxcYmV0YX0iLDFdLFsyLDMsIlxcZG93bmFycm93X1xcYmV0YSIsMV0sWzMsMSwiXFxjb25nIiwxLHsic3R5bGUiOnsidGFpbCI6eyJuYW1lIjoiYXJyb3doZWFkIn19fV0sWzIsMCwiXFxwc2kiLDEseyJvZmZzZXQiOi0yfV0sWzAsMiwiXFx0aGV0YSIsMSx7Im9mZnNldCI6LTJ9XV0=
\[\begin{tikzcd}
    {\boxed\lambda} && {\boxed{\vec \lambda}} \\
    \\
    {\boxed{\text{NF}}} && {\boxed{\text{NT}}}
    \arrow["\psi"{description}, shift left=2, from=1-1, to=1-3]
    \arrow["{\downarrow_\beta}"{description}, from=1-1, to=3-1]
    \arrow["\theta"{description}, shift left=2, from=1-3, to=1-1]
    \arrow["{\downarrow_{\beta}}"{description}, from=1-3, to=3-3]
    \arrow["\cong"{description}, <->, from=3-1, to=3-3]
  \end{tikzcd}\]

% The $\vec \beta$ refers to the $\beta$-reduction steps in system $\LamV$ and is used in the diagram to create a clear distinction.
In this diagram: the horizontal arrows symbolize the inverse maps underlying the isomorphism; the down arrows symbolize (partial) maps associating expressions to the respective beta-normal form (when existing).
Also recall the sets NF$\subset \Lam$-terms and NT$\subset \LamV$-terms of the respective $\beta$-normal forms.

\section{Mappings $\theta$ and $\psi$}

We start by defining the maps between expressions of $\Lam$ and $\LamV$ underlying the isomorphism.

\begin{definition}[Maps $\theta$ and $\theta'$]
  The map $\theta : \LamV \text{-terms} \to \Lam \text{-terms}$ is defined simultaneously with the map $\theta' : (\Lam \text{-terms} \times \LamV \text{-lists}) \to \Lam \text{-terms}$ by recursion on $\LamV$-terms and $\LamV$-lists respectively, as follows:
  \begin{center}
  \begin{tabular}{cc}
    $ \begin{aligned}[t]
      \theta(var(x)) &= x \\
      \theta(\lambda x . t) &= \lambda x . \theta(t) \\
      \theta(app_v(x, u, l)) &= \theta'(x, u::l) \\
      \theta(app_\lambda (x.t, u, l)) &= \theta'(\lambda x . \theta(t), u::l)
    \end{aligned} $
    & \quad
    $ \begin{aligned}[t] \\
      \theta'(M, [])   &= M \\
      \theta'(M, u::l) &= \theta'(M \ \theta(u), l).
    \end{aligned} $
  \end{tabular}
  \end{center}
\end{definition}
  
\begin{definition}Maps $\psi$ and $\psi'$
  The map $\psi' : (\Lam \text{-terms} \times \LamV \text{-lists}) \to \LamV \text{-terms}$ is defined by recursion on $\Lam$-terms as follows:
  \begin{align*}
    \psi(x, []) &= var(x) \\
    \psi(x, u::l) &= app_v (x, u, l) \\
    \psi(\lambda x . M, []) &= \lambda x . \psi(M) \\
    \psi(\lambda x . M, u::l) &= app_\lambda (x . \psi(M), u, l) \\
    \psi(M N, l) &= \psi'(M, \psi(N)::l),              
  \end{align*}
  where $\psi(M)$ is easily defined as $\psi'(M, [])$.
\end{definition}

\subsection{Bijection at the level of terms}

Now, we will establish that $\theta$ and $\psi$ are indeed inverse maps, and thus, $\Lam$-terms and $\LamV$-terms are in bijection.

\begin{lemma}
  \label{theta_inversion_lemma}
  \begin{align*}
    \theta \circ \psi' &= \theta'    
  \end{align*}
\end{lemma}
\begin{proof}
  The proof proceeds by induction on the structure of $\Lam$-terms and proper inspection of the $\LamV$-list in the variable and abstraction cases.
\end{proof}


\begin{theorem}[$\theta$ is left inverse of $\psi$]
  \label{theta_psi_inversion}
  \begin{align*}
    \theta \circ \psi &= id_{\Lam \text{-terms}}
  \end{align*}
\end{theorem}
\begin{proof}
  Immediate using~\cref{theta_inversion_lemma}.
\end{proof}


\begin{theorem}[$\psi$ is left inverse of $\theta$]
  \label{psi_theta_inversion}
  \begin{align*}
    \psi \circ \theta &= id_{\LamV \text{-terms}} \\
    \psi \circ \theta' &= \psi'    
  \end{align*}
\end{theorem}
\begin{proof}
  The proof proceeds by simultaneous induction on the structure of $\LamV$-terms and $\LamV$-lists, respectively.
\end{proof}


\subsection{Isomorphism at the level of reduction}

Now we turn our attention to reduction, showing that the reduction relations $\to_\beta$ of $\Lam$-calculus and $\LamV$ are isomorphic.

First, introduce some lemmata, in order to relate the mappings $\theta'$ and $\psi'$ with the @ operation and list append.

\begin{lemma}
  \label{theta_app_lemma}
  For every $\LamV$-terms $t, u$ and $\LamV$-list $l$,
  \[ \theta(t@(u, l)) = \theta'(\theta(t) \ \theta(u), l) \]
  and also, for every $\Lam$-term $M$, $\LamV$-term $u'$ and $\LamV$-lists $l, l'$,
  \[ \theta'(M, l+(u'::l')) = \theta'(\theta'(M, l) \ \theta(u'), l'). \]
\end{lemma}
\begin{proof}
  The proof proceeds easily by simultaneous induction on the structure of the $\LamV$-term $t$ and $\LamV$-list $l$, respectively.
\end{proof}

\begin{corollary}
  \label{psi_app_lemma}
  For every $\Lam$-term $M$, $\LamV$-term $u$ and $\LamV$-list $l$,
  \[ \psi'(M, u :: l) = \psi(M)@(u, l). \]
\end{corollary}
\begin{proof}
  The result follows as a corollary of~\cref{theta_app_lemma}, using~\cref{psi_theta_inversion} and~\cref{theta_inversion_lemma} to rewrite the left-hand side of the equality.
\end{proof}

Using the previous lemmas, the preservation of the substitution operations by $\theta$ and $\psi$ follow.

\begin{lemma}
  \label{theta_subst_lemma}
  For every $\LamV$-terms $t, u$,
  \[ \theta(t[x := u]) = \theta(t)[x := \theta(u)] \]
  and also, for every $\Lam$-term $M$,  $\LamV$-term $u$ and $\LamV$-list $l$,
  \[ \theta'(M[x := \theta(u)], l[x := u]) = \theta'(M, l)[x := u]. \]
\end{lemma}
\begin{proof}
  The proof follows by simultaneous induction on the structure of $t$ and $l$, using~\cref{theta_app_lemma}.
\end{proof}
  
\begin{lemma}
  \label{psi_subst_lemma}
  For every $\Lam$-terms $M, N$ and $\LamV$-list $l$,
  \[ \psi'(M[x := N], l[x := \psi(N)]) = \psi'(M, l)[x := \psi(N)]. \]
\end{lemma}
\begin{proof}
  The proof follows by induction on the structure of $\Lam$-term $M$, using~\cref{psi_app_lemma}.
\end{proof}

Now, we are essentially ready to obtain the isomorphism at the level of reduction.

\begin{lemma}
  \label{theta_step_lemma}
  For every $\Lam$-terms $M, N$ and $\LamV$-list $l$,
  \[ M \to_{\beta} N \implies \theta'(M, l) \to_{\beta} \theta'(N, l). \]
\end{lemma}
\begin{proof}
  The proof follows easily by induction on the structure of the $\LamV$-list $l$.
\end{proof}

\begin{theorem}[Preservation of reduction by $\theta$]
  \label{theta_step_pres}
  For every $\LamV$-terms $t, t'$,
  \[ t \to_{\beta} t' \implies \theta(t) \to_{\beta} \theta(t') \]
  and also, for every $\Lam$-term $M$ and $\LamV$-lists $l, l'$,
  \[ l \to_{\beta} l' \implies \theta'(M, l) \to_{\beta} \theta(M, l'). \]
\end{theorem}
\begin{proof}
  The proof proceeds by simultaneous induction on the structure of the step relation on $\LamV$-expressions.

  \cref{theta_app_lemma} is useful for the cases of compatibility steps.

  \cref{theta_subst_lemma} is crucial for cases dealing with $\beta$ steps.
\end{proof}


\begin{theorem}[Preservation of reduction by $\psi'$]
  \label{psi_step_pres}
  For every $\Lam$-terms $M, N$ and $\LamV$-list $l$,
  \[ M \to_{\beta} N \implies \psi'(M, l) \to_{\beta} \psi'(N, l). \]
\end{theorem}
\begin{proof}
  The proof proceeds by induction on the structure of the step relation on $\Lam$-terms.

  \cref{psi_subst_lemma} is crucial for cases dealing with $\beta$ steps.
\end{proof}


\subsection{Bijection at the level of normal forms}

For the following results recall both~\cref{beta_nfs} and~\cref{beta_nfs_can}.
We prove both bijections directly and give a hint on how they could be proved using the claims that we have provided.
One can find similar results to this in~\cite{LuisDychkoff}.

\begin{theorem}
  \begin{align*}
    \psi \circ \theta |_{\text{NT}} &= id_{\text{NT}} \\
    \psi \circ \theta'|_{\text{NA} \times \text{NL}} &= \psi' |_{\text{NA} \times \text{NL}}
  \end{align*}
\end{theorem}
\begin{proof}
  The proof proceeds easily by simultaneous induction on the structure of $\LamV$-terms in $\text{NT}$ and $\LamV$-lists in $\text{NL}$.
\end{proof}

\begin{theorem}
  \begin{align*}
    \theta \circ \psi |_{\text{NF}} &= id_{\text{NF}} \\
    \theta \circ \psi'|_{\text{NA} \times \text{NL}} &= \theta' |_{\text{NA} \times \text{NL}}
  \end{align*}
\end{theorem}
\begin{proof}
  The proof proceeds easily by simultaneous induction on the structure of $\Lam$-terms in $\text{NF}$ and $\text{NA}$.
\end{proof}

Alternatively, we could prove that $M \in \text{NF} \implies \psi(M) \in \text{NT}$ and $t \in \text{NT} \implies \theta(t) \in \text{NF}$. Then, the shown theorems would automatically follow.

For example, from the assumption that $M \in \text{NF}$, using~\cref{beta_nfs_claim}, one gets that $M$ is irreducible by $\to_\beta$.
Then, from~\cref{psi_step_pres}, $\psi(M)$ is also irreducible by $\to_\beta$ (in $\LamV$), which in turn means that $\psi(M) \in \text{NT}$ (by~\cref{beta_nfs_can_claim}).


\subsection{Isomorphism at the level of typed terms}

\begin{theorem}[$\theta$ admissibility]
  % \label{some_theorem_name}
  The following rules are admissible:
  \[ \begin{prooftree}
      \hypo{ \Gamma \vdash t : A }
      \infer1{ \Gamma \vdash \theta(t) : A } 
    \end{prooftree}
    \qquad \qquad
    \begin{prooftree}
      \hypo{ \Gamma \vdash M : A }
      \hypo{ \Gamma ; A \vdash l : B }
      \infer2{ \Gamma \vdash \theta'(M, l) : B }.
    \end{prooftree} \]
\end{theorem}
\begin{proof}
  The proof proceeds easily by simultaneous induction on the structure of the typing rules of $\LamV$-terms.
\end{proof}

\begin{theorem}[$\psi'$ admissibility]
  % \label{theorem12}
  The following rules is admissible:
  \[ \begin{prooftree}
      \hypo{ \Gamma \vdash M : A }
      \hypo{ \Gamma ; A \vdash l : B }
      \infer2{ \Gamma \vdash \psi'(M, l) : B }.
    \end{prooftree} \]
\end{theorem}
\begin{proof}
  The proof proceeds easily by induction on the structure of the typing rules of $\Lam$-terms.
\end{proof}

% ---

\section{Mechanisation in \textit{Rocq}}

In this section we provide a brief description of the mechanisations, as they follow from many previous definitions.
Essentially, we just defined maps $\theta$ and $\psi$ and mechanised every result.

One detail that may be highlighted is the definition for maps $\theta$ and $\theta'$.
\begin{lstlisting}[language=Coq]
Fixpoint theta (t: Canonical.term) : Lambda.term :=
  match t with
  | Vari x => Var x
  | Lamb t => Lam (theta t)
  | VariApp x u l => fold_left (fun s v => App s (theta v)) (u::l) (Var x)
  | LambApp t u l => fold_left (fun s v => App s (theta v)) (u::l) (Lam (theta t))
  end.

Definition theta' (s: Lambda.term) (l: list Canonical.term) :
  Lambda.term := fold_left (fun s v => App s (theta v)) l s.
\end{lstlisting}

The mechanised object that represents map $\theta'$ uses a higher order function on lists called \lst$fold_left$ that behaves exactly as $\theta'$, given the function \lst$(fun s v => App s (theta v))$ which folds the $\LamV$-list into a $\Lam$-term.

This was an undesired consequence of the use of polymorphic lists in the definition for $\LamV$-terms.
We could not define mutually recursive functions on the structure of the term and list because the proof assistant fails to recognise their termination.
Instead, we have to define these maps using higher order functions.
In this specific case, we could even enjoy the generality of the \lst$fold_left$ function.

As the mechanised $\theta'$ is defined after $\theta$, we have to consistently fold the definition for $\theta'$ in proofs to make them goal more readable.
This can be seen in the mechanisation of~\cref{theta_step_lemma}:
\begin{lstlisting}[language=Coq]
Lemma theta'_step_pres l :
  forall s s', Lambda.step s s' -> Lambda.step (theta' s l) (theta' s' l).
Proof.
  induction l as [| u l]; intros ; asimpl ; try easy.
  - fold (theta' (App s (theta u)) l).
    fold (theta' (App s' (theta u)) l).
    apply IHl. now constructor.
Qed.
\end{lstlisting}


%%% Local Variables:
%%% mode: LaTeX
%%% TeX-master: "../dissertation"
%%% End:
