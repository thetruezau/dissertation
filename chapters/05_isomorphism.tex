\chapter{The isomorphism $\Lam \cong \LamV$}
\label{c:isomorphism}
In~\cref{c:background}, the simply typed $\Lam$-calculus was introduced.

Now, we show an isomorphism between the system $\LamV$ introduced in the previous chapter and the simply typed $\Lam$-calculus.
This isomorphism will come at the level of syntax, reduction, typing rules and $\beta$-normal forms.

This is of great interest as $\LamV$ typing rules resemble a sequent calculus style.
Thus, we have a correspondence between natural deduction (typing rules of $\Lam$-calculus) and a fragment of sequent calculus.
The chapter is inspired in the works~\cite{LuisDychkoff} and~\cite[Chapter~4]{JCES2002}.
The following diagram summarizes what we will achieve in this chapter.

% https://q.uiver.app/#q=WzAsNCxbMiwwLCJcXGJveGVke1xcdmVjIFxcbGFtYmRhfSJdLFsyLDIsIlxcYm94ZWR7XFxzbWFsbHtcXHZlYyBcXGJldGEtbmZzfX0iXSxbMCwwLCJcXGJveGVkXFxsYW1iZGEiXSxbMCwyLCJcXGJveGVke1xcc21hbGx7XFxiZXRhLW5mc319Il0sWzAsMSwiXFxkb3duYXJyb3dfe1xcdmVjIFxcYmV0YX0iLDFdLFsyLDMsIlxcZG93bmFycm93X1xcYmV0YSIsMV0sWzIsMCwiXFxjb25nIiwxLHsic3R5bGUiOnsidGFpbCI6eyJuYW1lIjoiYXJyb3doZWFkIn19fV0sWzMsMSwiXFxjb25nIiwxLHsic3R5bGUiOnsidGFpbCI6eyJuYW1lIjoiYXJyb3doZWFkIn19fV1d
\[
  \begin{tikzcd}
    {\boxed \Lam} && {\boxed \LamV} \\
    \\
    {\boxed{\small{\beta-nfs}}} && {\boxed{\small{\vec \beta-nfs}}}
    \arrow["\cong"{description}, <->, from=1-1, to=1-3]
    \arrow["{\downarrow_\beta}"{description}, from=1-1, to=3-1]
    \arrow["{\downarrow_{\vec \beta}}"{description}, from=1-3, to=3-3]
    \arrow["\cong"{description}, <->, from=3-1, to=3-3]
  \end{tikzcd}
\]

\section{Mappings $\theta$ and $\psi$}

\begin{definition}
  Consider the following maps $\theta$ and $\theta'$:
  \begin{align*}
    \theta : \LamV \text{-terms} &\to \Lam \text{-terms} \\
    var(x) &\mapsto x \\
    \lambda x . t &\mapsto \lambda x . \theta(t) \\
    app_v (x, u, l) &\mapsto \theta'(x, u::l) \\
    app_\lambda (x. t, u, l) &\mapsto \theta'(\lambda x . \theta(t), u::l)
  \end{align*}
  \begin{align*}
    \theta' : (\Lam \text{-terms} \times \LamV \text{-lists}) & \to \Lam \text{-terms} \\
    (M, []) &\mapsto M \\
    (M, u::l) &\mapsto \theta'(M \ \theta(u), l).
  \end{align*}
\end{definition}
  
\begin{definition}
  Consider the following map $\psi'$:
  \begin{align*}
    \psi' : (\Lam \text{-terms} \times \LamV \text{-lists}) &\to \LamV \text{-terms} \\
    (x, []) &\mapsto var(x) \\
    (x, u::l) &\mapsto app_v (x, u, l) \\
    (\lambda x . M, []) &\mapsto \lambda x . \psi(M) \\
    (\lambda x . M, u::l) &\mapsto app_\lambda (x . \psi(M), u, l) \\
    (M N, l) &\mapsto \psi'(M, \psi(N)::l),              
  \end{align*}
  where $\psi(M)$ is defined as $\psi'(M, [])$.
\end{definition}

\subsection{Isomorphism at the level of terms}

\begin{lemma}
  \label{theta_inversion_lemma}
  \begin{align*}
    \theta \circ \psi' &= \theta'    
  \end{align*}
\end{lemma}
\begin{proof}
  The proof proceeds by induction on the structure of $\Lam$-terms.
\end{proof}


\begin{theorem}
  \label{theta_psi_inversion}
  \begin{align*}
    \theta \circ \psi &= id_{\Lam \text{-terms}}
  \end{align*}
\end{theorem}
\begin{proof}
  The proof proceeds by induction on the structure of $\Lam$-terms and uses as lemma for the application case the~\cref{inversion_lemma}.
\end{proof}


\begin{theorem}
  \label{psi_theta_inversion}
  \begin{align*}
    \psi \circ \theta &= id_{\LamV \text{-terms}} \\
    \psi \circ \theta' &= \psi'    
  \end{align*}
\end{theorem}
\begin{proof}
  The proof proceeds by simultaneous induction on the structure of $\LamV$-terms and $\LamV$-lists.
\end{proof}


\subsection{Isomorphism at the level of reduction}

First, we need to introduce some lemmata that establish the preservation of substitution operations by the mappings $\theta, \theta'$ and $\psi'$.
% Proofs of lemmas will now be omitted as they are all formalized in the proof assistant and usually proceed routinely.

\begin{lemma}
  \label{theta_app_lemma}
  For every $\LamV$-terms $t, u$ and $\LamV$-list $l$,
  \[ \theta(t@(u, l)) = \theta'(\theta(t) \ \theta(u), l) \]
  and also, for every $\Lam$-term $M$, $\LamV$-term $u'$ and $\LamV$-lists $l, l'$,
  \[ \theta'(M, l+(u'::l')) = \theta'(\theta'(M, l) \ \theta(u'), l'). \]
\end{lemma}
\begin{proof}
  The proof proceeds easily by simultaneous induction on the structure of the $\LamV$-term $t$ on the first proposition and on the structure of the $\LamV$-list $l$ on the second proposition.
\end{proof}

\begin{corollary}
  \label{psi_app_lemma}
  For every $\Lam$-term $M$, $\LamV$-term $u$ and $\LamV$-list $l$,
  \[ \psi'(M, u :: l) = \psi(M)@(u, l). \]
\end{corollary}
\begin{proof}
  The result follows as a corollary of~\cref{theta_app_lemma}, using~\cref{psi_theta_inversion} and~\cref{theta_inversion_lemma} to rewrite the left-hand side of the equality.
\end{proof}

\begin{lemma}
  \label{theta_subst_lemma}
  For every $\LamV$-terms $t, u$,
  \[ \theta(t[x := u]) = \theta(t)[x := \theta(u)] \]
  and also, for every $\Lam$-term $M$,  $\LamV$-term $u$ and $\LamV$-list $l$,
  \[ \theta'(M[x := \theta(u)], l[x := u]) = \theta'(M, l)[x := u]. \]
\end{lemma}
\begin{proof}
  The proof follows by simultaneous induction on the structure of $\LamV$-terms and -lists, using~\cref{theta_app_lemma}.
\end{proof}
  
\begin{lemma}
  \label{psi_subst_lemma}
  For every $\Lam$-terms $M, N$ and $\LamV$-list $l$,
  \[ \psi'(M[x := N], l[x := \psi(N)]) = \psi'(M, l)[x := \psi(N)]. \]
\end{lemma}
\begin{proof}
  The proof follows by simultaneous induction on the structure of $\LamV$-terms and -lists, using~\cref{psi_app_lemma}.
\end{proof}

Now, we can state the isomorphism at the level of reduction.

\begin{lemma}
  \label{theta_step_lemma}
  For every $\Lam$-terms $M, N$ and $\LamV$-list $l$,
  \[ M \to_{\beta} N \implies \theta'(M, l) \to_{\beta} \theta'(N, l). \]
\end{lemma}
\begin{proof}
  The proof follows easily by induction on the structure of the $\LamV$-list $l$.
\end{proof}

\begin{theorem}
  \label{theorem9}
  For every $\LamV$-terms $t, t'$,
  \[ t \to_{\beta} t' \implies \theta(t) \to_{\beta} \theta(t') \]
  and also, for every $\Lam$-term $M$ and $\LamV$-lists $l, l'$,
  \[ l \to_{\beta} l' \implies \theta'(M, l) \to_{\beta} \theta(M, l'). \]
\end{theorem}
\begin{proof}
  The proof proceeds by simultaneous induction on the structure of the step relation on $\LamV$-terms.

  \cref{theta_app_lemma} is useful for the cases of compatibility steps.

  \cref{theta_subst_lemma} is crucial for cases dealing with $\beta$ steps.
\end{proof}


\begin{theorem}
  \label{theorem10}
  For every $\Lam$-terms $M, N$ and $\LamV$-list $l$,
  \[ M \to_{\beta} N \implies \psi'(M, l) \to_{\beta} \psi(N, l). \]
\end{theorem}
\begin{proof}
  The proof proceeds by simultaneous induction on the structure if the step relation on $\Lam$-terms.

  \cref{psi_subst_lemma} is crucial for cases dealing with $\beta$ steps.
\end{proof}

\subsection{Isomorphism at the level of typed terms}

\begin{theorem}[$\theta$ admissibility]
  \label{theorem11}
  The following rules are admissible:
  \[ \begin{prooftree}
      \hypo{ \Gamma \vdash t : A }
      \infer1{ \Gamma \vdash \theta(t) : A } 
    \end{prooftree}
    \qquad \qquad
    \begin{prooftree}
      \hypo{ \Gamma \vdash M : A }
      \hypo{ \Gamma ; A \vdash l : B }
      \infer2{ \Gamma \vdash \theta'(M, l) : B }.
    \end{prooftree} \]
\end{theorem}
\begin{proof}
  The proof proceeds easily by simultaneous induction on the structure of the typing rules of $\LamV$-terms.
\end{proof}

\begin{theorem}[$\psi'$ admissibility]
  \label{theorem12}
  The following rules is admissible:
  \[ \begin{prooftree}
      \hypo{ \Gamma \vdash M : A }
      \hypo{ \Gamma ; A \vdash l : B }
      \infer2{ \Gamma \vdash \psi'(M, l) : B }.
    \end{prooftree} \]
\end{theorem}
\begin{proof}
  The proof proceeds easily by induction on the structure of the typing rules of $\Lam$-terms.
\end{proof}

\subsection{Isomorphism at the level of normal forms}



\section{Mechanisation in \textit{Rocq}}

\subsection{\lst$LambdaIsomorphism.v$}
\subsection{\lst$NormalIsomorphism.v$}


%%% Local Variables:
%%% mode: LaTeX
%%% TeX-master: "../dissertation"
%%% End:
