\chapter{Conclusions}
\label{c:conclusions}

In this chapter, we first describe our contributions and then discuss possible directions for future work.

\section{Contributions}

We list below the contributions achieved with this dissertation.

First and more important, we used the \textit{Rocq Prover} and the \textit{Autosubst} library to mechanise the following systems introduced in this work:
\begin{enumerate}
\item the multiary $\Lam$-calculus (system $\LamM$);
\item the canonical subsystem of $\LamM$;
\item the canonical $\Lam$-calculus (system $\LamV$).
\end{enumerate}
Taking the formalisations of these systems on the proof assistant, we also obtained computer verified proofs for results such as:
\begin{enumerate}
\item subject reduction for systems $\LamM$ and $\LamV$;
\item isomorphism between the canonical subsystem of $\LamM$ and system $\LamV$;
\item conservativeness of $\LamV$ over $\LamM$;
\item isomorphism between the simply typed $\Lam$-calculus and system $\LamV$.
\end{enumerate}

Second, we gave an exhaustive definition for the concept of subsystem, separating two isomorphic representations of the canonical subsystem of $\LamM$.
This helped us clarify the loose idea of subsystem and simplify some of the result proven (for example, using the self-contained system $\LamV$ for proving the theorem of conservativeness).
From this idea, we could even propose a standard aproach to formalise any subsystem.

Third and last, through this document, a detailed exposition of the mechanised systems and proofs using the \textit{Rocq Prover} along with some digressions over formalisation choices.

% Fourth and last, we layed out a modularised formalisation that helped us enjoy the automated tactics of the \textit{Rocq Prover}.
% This way, many of the provided proofs were obtained automatically.

\section{Future work}

% We did not focus in results of confluency that could easily be added
In this dissertation, we did not focus so much on results such as the confluency of reduction in our systems.
We suspect that these could easily be added with the help of the \textit{Autosubst} library (one can found a case study of confluence of reduction in \cite[Section~2.2]{AutosubstSchafer}).

Furthermore,
\begin{itemize}
\item Could we have more automation?
  \subitem Could we only provide proofs for the hard cases proved in paper (easy results should be done automatically)?
  \subitem Could we have more \textit{Autosubst} automation available for our systems? Discuss this here? What about \textit{Autosubst2}?
  \subitem Could we have simpler formalisations? How so?
\item System $\pmb{\lambda Jm}$ is still not formalised.
\item Our modular definitions allow us to use the \textit{Autosubst} library to enrich our typing systems (dependent types, intersection types, SystemF, Lambda Cube and so on).

  % ... \item Many more mechanisations of metatheory?
\end{itemize}

\begin{comment}
Mechanisations in relation with the formalisations on the paper.
\begin{itemize}
\item Some ideas for the metatheory formalised come from attempts of mechanisations.
\item The metatheory mechanised does not correspond exactly to the formalised in the literature.
  \subitem An example: the polymorphic definition for system  $\LamM$.
\end{itemize}

Further work?
\begin{itemize}
\item Because of the modularity and of the \textit{Autosubst} library we have the facility to enrich our typing systems (ex: SystemF?).
\item Could there be more automation?  
\item Why use a outdated library for mechanising binders? What about \textit{Autosubst 2}?
\item Obstacles on the mechanisation of a non standard substitution operation.
\item SSreflect style proofs for \textit{Rocq Prover}.
\end{itemize}
\end{comment}
