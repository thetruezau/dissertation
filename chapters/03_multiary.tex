\chapter{The Multiary Lambda Calculus and Its Canonical Fragment }
\label{c:multiary}

\section{The Multiary Lambda Calculus ($\lambda m$)}

\begin{definition}[$\lambda m$-terms]
  The $\lambda m$-terms are defined by the following grammar:
  
  \begin{align*} 
    t, u \ &::= \ x \ | \ \lambda x . t \ | \ t(u, l) \ \\
    l      &::= \ []\  | \ u :: l
  \end{align*}
\end{definition}

\begin{definition}[Concatenation of $\lambda m$-lists]
  The concatenation of two $\lambda m$-lists, $l + l'$ is defined as follows:

  \begin{enumerate}
  \item $[] + l' = l'$
  \item $(u::l) + l' = u::(l + l')$.
  \end{enumerate}
\end{definition}

% a seguinte definicao e estranha mas vamos seguir

\begin{definition}[Substitution for $\lambda m$-terms]
  The substitution over a $\lambda m$-term is mutually defined with the substitution over a $\lambda m$-list as follows:
  
  \begin{enumerate}
  \item $x[x := v] = v$
  \item $y[x := v] = y \text{, with } x \neq y$
  \item $(\lambda y . t)[x := v] = \lambda y . (t[x := v])$
  \item $t(u, l)[x := v] = t[x := v](u[x := v], l[x := v])$

  \item $([])[x := v] = []$
  \item $(u::l)[x := v] = u[x := v] :: l[x := v]$.
  \end{enumerate}
\end{definition}

\begin{definition}[Reduction rules for $\lambda m$-terms]  
  \[
    (\lambda x . t)(u, []) \to t[x := u]
    \label{eq:ruleb1} \tag{$\beta_1$}
  \]
  \[
    (\lambda x . t)(u, v::l) \to t[x := u](v, l)
    \label{eq:ruleb2} \tag{$\beta_2$}
  \]
  \[
    t(u, l)(u', l') \to t(u, l + (u'::l'))
    \label{eq:ruleh} \tag{$h$}
  \]
\end{definition}

\begin{definition}[Typing Rules for $\lambda m$-terms]
  \[
    \begin{prooftree}
      \infer0[Var]{ x:A, \Gamma \vdash x:A } 
    \end{prooftree}
    \qquad
    \begin{prooftree}
      \hypo{ x:A, \Gamma \vdash t:B }
      \infer1[Abs]{ \Gamma \vdash \lambda x . t : A \supset B  } 
    \end{prooftree}
  \]
  \[
    \begin{prooftree}
      \hypo{ \Gamma \vdash t : A \supset B }
      \hypo{ \Gamma \vdash u : A }
      \hypo{ \Gamma ; B \vdash l : C }	
      \infer3[mApp]{ \Gamma \vdash t(u, l) : C } 
    \end{prooftree}
  \]
  \[
    \begin{prooftree}
      \infer0[Nil]{ \Gamma ; A \vdash []:A } 
    \end{prooftree}
    \qquad
    \begin{prooftree}
      \hypo{ \Gamma \vdash u:A }
      \hypo{ \Gamma ; B \vdash l:C }
      \infer2[Cons]{ \Gamma ; A \supset B \vdash  u::l : C } 
    \end{prooftree}
  \]
\end{definition}

\section{The Canonical Fragment ($\vec \lambda m$)}

% Syntax and Typification?

\begin{definition}[$\vec \lambda m$-terms]
  The $\vec \lambda m$-terms are defined by the following grammar:
  
  \begin{align*} 
    t, u \ &::= \ x \ | \ \lambda x . t \ | \ x(u, l) \ | \ (\lambda x . t)(u, l) \\
    l      &::= \ []\  | \ u :: l
  \end{align*}
\end{definition}

\begin{definition}[Substitution for $\lambda m$-terms]
  The substitution over a $\vec \lambda m$-term is mutually defined with the substitution over a $\vec \lambda m$-list as follows:
  
  \begin{enumerate}
  \item $x[x := v] = v$
  \item $y[x := v] = y \text{, with } x \neq y$
  \item $(\lambda y . t)[x := v] = \lambda y . (t[x := v])$
  \item $x(u, l)[x := v] = v @ (u[x := v], l[x := v])$
  \item $y(u, l)[x := v] = y(u[x := v], l[x := v]) \text{, with } x \neq y$
  \item $(\lambda y . t)(u, l)[x := v] = (\lambda y . t[x := v])(u[x := v], l[x := v])$    
    
  \item $([])[x := v] = []$
  \item $(u::l)[x := v] = u[x := v] :: l[x := v]$.
  \end{enumerate}
\end{definition}

\begin{definition}[Typing Rules for $\vec \lambda m$-terms]
  \[
    \begin{prooftree}
      \infer0[Var]{ x:A, \Gamma \vdash x:A } 
    \end{prooftree}
    \qquad
    \begin{prooftree}
      \hypo{ x:A, \Gamma \vdash t:B }
      \infer1[Abs]{ \Gamma \vdash \lambda x . t : A \supset B  } 
    \end{prooftree}
  \]
  \[
    \begin{prooftree}
      \hypo{ \Gamma, x:A \supset B \vdash u:A}
      \hypo{ \Gamma, x:A \supset B ; B \vdash l:C }	
      \infer2[VarApp]{ \Gamma, x:A \supset B \vdash x(u, l) : C } 
    \end{prooftree}
  \]
  \[
    \begin{prooftree}
      \hypo{ \Gamma, x:A \vdash t:B }
      \hypo{ \Gamma \vdash u:A }
      \hypo{ \Gamma ; B \vdash l : C }	
      \infer3[LamApp]{ \Gamma \vdash (\lambda x . t)(u, l) : C } 
    \end{prooftree}
  \]
  \[
    \begin{prooftree}
      \infer0[Nil]{ \Gamma ; A \vdash []:A } 
    \end{prooftree}
    \qquad
    \begin{prooftree}
      \hypo{ \Gamma \vdash u:A }
      \hypo{ \Gamma ; B \vdash l:C }
      \infer2[Cons]{ \Gamma ; A \supset B \vdash  u::l : C } 
    \end{prooftree}
  \]
\end{definition}

\section{Formalised Results}

\begin{theorem}[Subject Reduction]
\end{theorem}

\begin{theorem}[Conservativeness]
\end{theorem}

\section{Comments on the Formalisation}
\subsection{A Nested Inductive Type}
\subsection{Formalising a Subsystem}

%%% Local Variables:
%%% mode: LaTeX
%%% TeX-master: "../dissertation"
%%% End:
