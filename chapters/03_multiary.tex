\chapter{The Multiary Lambda Calculus and Its Canonical Fragment }
\label{c:multiary}

\section{The Multiary Lambda Calculus ($\lambda m$)}

\begin{definition}[$\lambda m$-terms]
  The $\lambda m$-terms are defined by the following grammar:  
  \begin{align*} 
    t, u \ &::= \ x \ | \ \lambda x . t \ | \ t(u, l) \ \\
    l      &::= \ []\  | \ u :: l
  \end{align*}
\end{definition}

\begin{definition}[Concatenation of $\lambda m$-lists]
  The concatenation of two $\lambda m$-lists, $l + l'$, is defined as follows:
  \begin{align*}
  & [] + l' = l', \\
  & (u::l) + l' = u::(l + l').
  \end{align*}
\end{definition}

\begin{definition}[Substitution for $\lambda m$-terms]
  The substitution over a $\lambda m$-term is mutually defined with the substitution over a $\lambda m$-list as follows:  
  \begin{align*}
  & x[x := v] = v ; \\
  & y[x := v] = ya \text{, with } x \neq y ; \\
  & (\lambda y . t)[x := v] = \lambda y . (t[x := v]) ; \\
  & t(u, l)[x := v] = t[x := v](u[x := v], l[x := v]) ; \\    
  & ([])[x := v] = [] ; \\ 
  & (u::l)[x := v] = u[x := v] :: l[x := v] .
  \end{align*}
\end{definition}

\begin{definition}[Reduction rules for $\lambda m$-terms]  
  \[
    (\lambda x . t)(u, []) \to t[x := u]
    \label{eq:ruleb1} \tag{$\beta_1$}
  \]
  \[
    (\lambda x . t)(u, v::l) \to t[x := u](v, l)
    \label{eq:ruleb2} \tag{$\beta_2$}
  \]
  \[
    t(u, l)(u', l') \to t(u, l + (u'::l'))
    \label{eq:ruleh} \tag{$h$}
  \]
\end{definition}

\begin{definition}[Compatible Relation]
  Let $R$ and $R'$ be two binary relations on $\lambda m$-terms and $\lambda m$-lists respectively.
  We say they are compatible when they satisfy:
  \[
    \begin{prooftree}
      \hypo{ (t, t') \in R }
      \infer1{ (\lambda x . t, \lambda x . t') \in R } 
    \end{prooftree}
    \quad \ \
    \begin{prooftree}
      \hypo{ (t, t') \in R }
      \infer1{ (t(u, l), t'(u, l)) \in R } 
    \end{prooftree}
    \quad \ \
    \begin{prooftree}
      \hypo{ (u, u') \in R }
      \infer1{ (t(u, l), t(u', l)) \in R } 
    \end{prooftree}
    \quad \ \
    \begin{prooftree}
      \hypo{ (l, l') \in R' }
      \infer1{ (t(u, l), t(u, l')) \in R } 
    \end{prooftree}
  \]
  \[
    \begin{prooftree}
      \hypo{ (u, u') \in R }
      \infer1{ (u::l, u'::l) \in R' } 
    \end{prooftree}
    \qquad
    \begin{prooftree}
      \hypo{ (l, l') \in R' }
      \infer1{ (u::l, u::l') \in R' } 
    \end{prooftree}
  \]
\end{definition}


\begin{definition}[Typing Rules for $\lambda m$-terms]
  \[
    \begin{prooftree}
      \infer0[Var]{ x:A, \Gamma \vdash x:A } 
    \end{prooftree}
    \qquad
    \begin{prooftree}
      \hypo{ x:A, \Gamma \vdash t:B }
      \infer1[Abs]{ \Gamma \vdash \lambda x . t : A \supset B  } 
    \end{prooftree}
  \]
  \[
    \begin{prooftree}
      \hypo{ \Gamma \vdash t : A \supset B }
      \hypo{ \Gamma \vdash u : A }
      \hypo{ \Gamma ; B \vdash l : C }	
      \infer3[mApp]{ \Gamma \vdash t(u, l) : C } 
    \end{prooftree}
  \]
  \[
    \begin{prooftree}
      \infer0[Nil]{ \Gamma ; A \vdash []:A } 
    \end{prooftree}
    \qquad
    \begin{prooftree}
      \hypo{ \Gamma \vdash u:A }
      \hypo{ \Gamma ; B \vdash l:C }
      \infer2[Cons]{ \Gamma ; A \supset B \vdash  u::l : C } 
    \end{prooftree}
  \]
\end{definition}

\section{The Canonical Fragment ($\vec \lambda m$)}

% Syntax and Typification?

\begin{definition}[$\vec \lambda m$-terms]
  The $\vec \lambda m$-terms are defined by the following grammar:
  
  \begin{align*} 
    t, u \ &::= \ x \ | \ \lambda x . t \ | \ x(u, l) \ | \ (\lambda x . t)(u, l) \\
    l      &::= \ []\  | \ u :: l
  \end{align*}
\end{definition}

\begin{definition}[Substitution for $\lambda m$-terms]
  The substitution over a $\vec \lambda m$-term is mutually defined with the substitution over a $\vec \lambda m$-list as follows:
  
  \begin{align*}
  & x[x := v] = v ; \\
  & y[x := v] = y \text{, with } x \neq y ; \\
  & (\lambda y . t)[x := v] = \lambda y . (t[x := v]) ; \\
  & x(u, l)[x := v] = v @ (u[x := v], l[x := v]) ; \\
  & y(u, l)[x := v] = y(u[x := v], l[x := v]) \text{, with } x \neq y ; \\
  & (\lambda y . t)(u, l)[x := v] = (\lambda y . t[x := v])(u[x := v], l[x := v]) ; \\
  & ([])[x := v] = [] ; \\
  & (u::l)[x := v] = u[x := v] :: l[x := v] .
  \end{align*}
\end{definition}

\begin{definition}[Typing Rules for $\vec \lambda m$-terms]
  \[
    \begin{prooftree}
      \infer0[Var]{ x:A, \Gamma \vdash x:A } 
    \end{prooftree}
    \qquad
    \begin{prooftree}
      \hypo{ x:A, \Gamma \vdash t:B }
      \infer1[Abs]{ \Gamma \vdash \lambda x . t : A \supset B  } 
    \end{prooftree}
  \]
  \[
    \begin{prooftree}
      \hypo{ \Gamma, x:A \supset B \vdash u:A}
      \hypo{ \Gamma, x:A \supset B ; B \vdash l:C }	
      \infer2[VarApp]{ \Gamma, x:A \supset B \vdash x(u, l) : C } 
    \end{prooftree}
  \]
  \[
    \begin{prooftree}
      \hypo{ \Gamma, x:A \vdash t:B }
      \hypo{ \Gamma \vdash u:A }
      \hypo{ \Gamma ; B \vdash l : C }	
      \infer3[LamApp]{ \Gamma \vdash (\lambda x . t)(u, l) : C } 
    \end{prooftree}
  \]
  \[
    \begin{prooftree}
      \infer0[Nil]{ \Gamma ; A \vdash []:A } 
    \end{prooftree}
    \qquad
    \begin{prooftree}
      \hypo{ \Gamma \vdash u:A }
      \hypo{ \Gamma ; B \vdash l:C }
      \infer2[Cons]{ \Gamma ; A \supset B \vdash  u::l : C } 
    \end{prooftree}
  \]
\end{definition}

\section{Formalised Results}

\begin{theorem}[Subject Reduction]
  Given $\lambda m$-terms $t$ and $u$, we have the follwing:
  \[
    \Gamma \vdash t : A \implies \Gamma \vdash t[x := u] : A
  \]
\end{theorem}

\begin{theorem}[Conservativeness]
  Given $\vec \lambda m$-terms $t$ and $t'$, we have the follwing:
  \[
    t \to^{*}_{\vec \lambda m} t' \Longleftrightarrow
    t \to^{*}_{\lambda m} t'
  \]
\end{theorem}

\section{Comments on the Formalisation}

\subsection{A Nested Inductive Type}
\begin{itemize}
\item AUTOSUBST excavation for support
\item induction principle and further proofs
\item generalization or specification?
\end{itemize}

\subsection{Formalising a Subsystem}

\begin{itemize}
\item Carrgying a predicate
\item Subset types in Coq
\item A self contained representation
\end{itemize}

%%% Local Variables:
%%% mode: LaTeX
%%% TeX-master: "../dissertation"
%%% End:
