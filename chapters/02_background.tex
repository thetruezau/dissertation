\chapter{Background}
\label{c:background}

% In this chapter, we intend to take the example of formalising some theory about the $\Lam$-calculus as a motivation to display some core aspects involved in the more elaborate formalisations ahead.

The following chapter introduces essential background to aid the reading of this dissertation.
% It will take the scheme used in the subsequent chapters:
% a first part of theoretical presentation and a second part of mechanisation.
First, we introduce the well-known simply typed $\Lam$-calculus.
Then, we delve into the theory on mechanisation of meta-theory, specifically in the context of our work.
These concepts are introduced and motivated by the task of formalising the $\Lam$-calculus system introduced.

\section{Simply typed $\Lam$-calculus}

For the untyped lambda calculus descriptions we refer to \cite{Barendregt1987}.
For what types and the simply typed lambda calculus is about we refer to \cite{Barendregt2013} and \cite{Hindley1997}.

\subsection{Syntax}

\begin{definition}[$\Lam$-terms]
  The $\Lam$-terms are defined by the following grammar:
  \[ M, N \ ::= \ x \ | \ (\lambda x . M) \ | \ (M N) \]
  where $x$ denotes any variable, typically in the range of $x, y, z$.
\end{definition}

\begin{notation}
  We shall assume the usual notation conventions on $\Lam$-terms:

  \begin{enumerate}
  \item Outermost parenthesis are omitted.
  \item Multiple abstractions can be abreviated as $\lambda x y z . M$ instead of  $\lambda x . (\lambda y . (\lambda z . M))$.
  \item Multiple applications can be abreviated as $M N_1 N_2$ instead of $(M N_1) N_2$.
  \end{enumerate}
\end{notation}

\begin{definition}[Free variables]
  For every $\Lam$-term $M$, we recursively define the set of free variables in $M$, $FV(M)$, as follows:  
  \begin{align*}
    & FV( x ) = \{ x \}, \\
    & FV( \lambda x . M ) = FV(M) - \{ x \}, \\
    & FV( M N ) = FV(M) \cup FV(N).
  \end{align*}

  When a variable occurring in a term is not free it is said to be bound.
\end{definition}

% \begin{remark}
%   Informally, abstractions will behave as functions.
%   As so, we do not care about the names of bound variables. 
%   This idea is formally introduced next.
% \end{remark}

\begin{definition}[$\alpha$-equality]
  We say that two $\Lam$-terms are $\alpha$-equal when they only differ in the name of their bound variables.
\end{definition}

\begin{remark}
  The previous informal definition lets us take advantage of a variable naming convention.
  With this notion of $\alpha$-equality, the definition of substitution over $\Lam$-terms and meta-discussion of our syntax will be simplified.
  % After defining the substitution operation we may introduce a better and more formal definition of the $\alpha$-conversion.
  After defining the substitution operation we will rigorously introduce the definition for $\alpha$-conversion.
\end{remark}

\begin{convention} 
  We will use the \textit{variable convention} introduced in \cite{Barendregt1987}.
  Every $\Lam$-term that we refer from now on is chosen (via $\alpha$-equality) to have bound variables with different names from free variables.
\end{convention}

\begin{definition}[Substitution]
  For every $\Lam$-term $M$, we recursively define the substitution of the free variable $x$ by $N$ in $M$, $M[x := N]$, as follows:
  \begin{align*}
    & x[x := N] = N; \\
    & y[x := N] = y \text{, with } x \neq y; \\
    & (\lambda y . M_1)[x := N] = \lambda y . (M_1[x := N]); \\
    & (M_1 M_2)[x := N] = (M_1[x := N]) (M_2[x := N]).
  \end{align*}
\end{definition}

\begin{remark}
  Is is important to notice that by variable convention, the substitution operation described is capture-avoiding
  - bound variables will not be substituted ($x \in FV(M)$) and the free variables in $N$ will not be affected by the binders in $M$, as they are chosen to have different names. 
\end{remark}

\begin{definition}[Compatible Relation]
  Let $R$ be a binary relation on $\Lam$-terms.
  We say that $R$ is compatible if it satisfies:
  \[
    \begin{prooftree}
      \hypo{ (M_1, M_2) \in R }
      \infer1{ (\lambda x . M_1, \lambda x . M_2) \in R } 
    \end{prooftree}
    \qquad
    \begin{prooftree}
      \hypo{ (M_1, M_2) \in R }
      \infer1{ (N M_1, N M_2) \in R } 
    \end{prooftree}
    \qquad
    \begin{prooftree}
      \hypo{ (M_1, M_2) \in R }
      \infer1{ (M_1 N, M_2 N) \in R }
    \end{prooftree}
  \]
\end{definition}


\begin{notation}
  Given a binary relation $R$ on $\Lam$-terms, we define:
  \begin{align*}
    & \to_R \text{as the compatible closure of $R$} ; \\
    & \twoheadrightarrow_R \text{as the reflexive and transitive closure of $\to_R$} ; \\
    & =_R \text{as the equivalence relation generated by $\twoheadrightarrow_R$}.
  \end{align*}
\end{notation}


\begin{definition}[$\alpha$-conversion]
  Consider the following binary relation on $\Lam$-terms:  
  \[
    \alpha = \{ (\lambda x . M, \lambda y . M[x := y]) \
                | \ \text{for every $y$ not occurring in $M$} \}.
  \]  
  We call $\alpha$-conversion to the generated $=_\alpha$ relation.
\end{definition}


\begin{definition}[$\beta$-reduction]
  Consider the following binary relation on $\Lam$-terms:  
  \[
    \beta = \{ ((\lambda x . M) N, M[x := N]) \
                | \ \text{for every $M, N$} \}.
  \]  
  We call one step $\beta$-reduction to the relation $\to_\beta$ and multistep $\beta$-reduction to the relation $\twoheadrightarrow_\beta$.
\end{definition}


\begin{definition}[$\beta$-normal forms]
  We inductively define the set of $\Lam$-terms in $\beta$-normal form, NF, and normal applications, NA, as follows:
  \[
    \begin{prooftree}
      \infer0{ x \in \text{NA} } 
    \end{prooftree}
    \qquad
    \begin{prooftree}
      \hypo{ M_1 \in \text{NA} }
      \hypo{ M_2 \in \text{NF} }            
      \infer2{ M_1 M_2 \in \text{NA} } 
    \end{prooftree}
    \qquad
    \begin{prooftree}
      \hypo{ M \in \text{NA} }
      \infer1{ M \in \text{NF} } 
    \end{prooftree}
    \qquad
    \begin{prooftree}
      \hypo{ M \in \text{NF} }
      \infer1{ \lambda x . M \in \text{NF} } 
    \end{prooftree}
  \]
  These $\Lam$-terms are irreducible according to $\to_\beta$.
\end{definition}

% --- 

\subsection{Types}

\begin{definition}[Simple Types]
  The simple types are defined by the following grammar:  
  \[
    A, B, C ::= p \ | \ (A \supset B)
  \]
  where $p$ denotes any atomic variable, typically in the range of $p, q, r$.
\end{definition}

\begin{notation}
  We will assume the usual notation conventions on simple types. 
  \begin{enumerate}
  \item Outermost parenthesis are omitted.
  \item Types associate to the right. Therefore, the type $A \supset (B \supset C)$ may often be written simply as $A \supset B \supset C$.
  \end{enumerate}
\end{notation}

\begin{definition}[Context]
  A context, $\Gamma, \Delta, \dots$, is a partial function from the variables of $\Lam$-terms to simple types.
\end{definition}

\begin{notation} \hfill
  \begin{enumerate}
  \item We may often refer to the partial function of as the set of pairs $(x, A)$ written as $x:A$.
  \item We will also simplify the set notation of contexts as follows:
    \begin{align*}
      &\mapsto \{ \} \\
      x:A         &\mapsto \{ x:A \} \\
      x:A, \Gamma &\mapsto \{ x:A \} \cup \Gamma
    \end{align*}
  \end{enumerate}
\end{notation}

\begin{definition}[Typing Rules for $\Lam$-terms]
  A type-assignment or sequent is a triple, $\Gamma \vdash M:A$, that is inductively defined by the following inference rules (or typing rules):
  \[
    \begin{prooftree}
      \infer0[Var]{ x:A, \Gamma \vdash x:A } 
    \end{prooftree}
    \qquad
    \begin{prooftree}
      \hypo{ x:A, \Gamma \vdash M:B }
      \infer1[Abs]{ \Gamma \vdash \lambda x . M : A \supset B  } 
    \end{prooftree}
    \qquad
    \begin{prooftree}
      \hypo{ \Gamma \vdash M : A \supset B }
      \hypo{ \Gamma \vdash N : A }	
      \infer2[App]{ \Gamma \vdash M N : B } 
    \end{prooftree}
  \]
\end{definition}

% ---

\section{Mechanising meta-theory in \textit{Rocq}}

Having introduced the ordinary $\Lam$-calculus, we take it as our goal of formalisation.
This helps motivating the main decisions behind our mechanisations.
The variations of $\Lam$-calculus that we are going to introduce will follow closely the approach described here with the corresponding adaptions.

% The mechanisation done is dependent on the theory provided by the \textit{Rocq Prover} - the Calculus of Inductive Constructions.
% We will follow assuming a basic knowledge on \textit{Rocq} and its syntax. % to define inductive types and proof techniques.

\subsection{The \textit{Rocq Prover}}

% For what refers to the \textit{Rocq Prover} (former \textit{Coq Proof Assistant}) we refer to \cite{CoqArt} \cite{RocqManual}.

% (calculus of inductive constructions)

The \textit{Rocq Prover} is an interactive theorem prover.
This is a tool that helps in the formalisation of mathematical results and that can generate machine-verified proofs via interaction with a human.

\dots

% (Check Kathrin Stark introduction)

% (Exemplo do tipo inductivo para inteiros?)

% (Tipos mutuamente inductivos? Combined Schemes?)

\subsection{Syntax with binders}

% If we were to formalise such a system like the $\Lam$-calculus introduced above, we would probably create an inductive type like the following, in \textit{Coq}.
Formalising the untyped $\Lam$-calculus syntax in \textit{Rocq} would result in an inductive definition similar to:

\begin{lstlisting}[language=Coq]
  Inductive term : Type :=
  | Var (x: var)
  | Lam (x: var) (t: term)
  | App (s: term) (t: term).
\end{lstlisting}

The question that every similar definition imposes is the definition of the \lst$var$ type. Following the usual pen-and-paper approach, this type would be a subset of a string type, where a variable is just a placeholder for a name.

Of course this is fine when dealing with proofs and definitions in a paper.
To simplify this, we can even take advantage of conventions, like the one referenced above (by Barendregt).
However, this variable definition can get rather exhausting  when it comes to rigorously define all this syntactical aspects and substitution operations.

There are several alternatives described in the literature of mechanisation of meta-theory. 
% This topic of binding was \sout{even} proposed in the POPLmark challenge \cite{POPLmark} as a way to discuss the potential of proof assistants.
The POPLmark challenge \cite{POPLmark} points to the topic of binding as central for discussing the potential of modern-day proof assistants.
From the many alternatives, we chose to focus in the nameless syntax proposed by de~Bruijn.

% Autosubst ? not yet
% POPl Mark
% falar de substituicoes ???

\subsection{De~Bruijn syntax}

In the 1970s, de Bruijn started working on the \textit{Automath} proof assistant and proposed a simplified syntax to deal with generic binders \cite{deBruijn}.
This approach is claimed to be good for meta-lingual discussion and for the computer and computer programme. In contrast, this syntax is further away from the human reader.

The main idea is to treat variables as indices (represented by natural numbers) and to interpret these indices as the distance to the respective binder.
Therefore, we will call these terms nameless. 
% $\Lam$-terms the nameless $\Lam$-terms.

\begin{definition}[nameless $\Lam$-terms]
  The nameless $\Lam$-terms are defined by the following grammar:

  \[ M, N \ ::= \ i \ | \ \lambda . M \ | \ M N \]

  where $i$ ranges over the natural numbers.
\end{definition}

\begin{remark}
  Nameless $\Lam$-terms have no $\alpha$-conversion since there is no freedom to choose the names of bound variables.
\end{remark}

Now, we will construct a different formulation for the concept of substitution.

\begin{definition}[Substitution]
  A substitution over nameless $\Lam$-terms is a function mapping natural numbers (indices) to $\Lam$-terms.
\end{definition}

Here are some examples of useful substitutions:
\begin{align*}
  id(k) &= k \\
  \uparrow(k) &= (k+1); \\
  (M \cdot \sigma)(k) &=
                        \begin{cases}
                          M & \ \text{if $k = 0$} \\
                          \sigma(k-1) & \ \text{if $k > 0$}.
                        \end{cases}
\end{align*}

\begin{definition}[Subsitution instantiation]
  The operation of instantiating a substitution $\sigma$ over a nameless $\Lam$-term $M$, $M[\sigma]$, is defined recursively by the following equations:
  \begin{align*}
    & i[\sigma] = \sigma(i); \\
    & (\lambda . M)[\sigma] = \lambda . (M[0 \cdot (\uparrow \circ \ \sigma)]); \\
    & (M_1 M_2)[\sigma] = (M_1[\sigma]) (M_2[\sigma]).
  \end{align*}
\end{definition}

\subsection{Autosubst library}

% \cite{AutosubstSchafer}

The \textit{Autosubst} library for the \textit{Rocq Prover} simplifies the formalisation of syntax with binders.
% This is obtained using the theory of explicit substitutions to fragment and clarify the description of substitution operations.
It provides the \textit{Rocq Prover} with tactics to define substitution over an inductively defined syntax.
Furthermore, it even offers some automation for proofs dealing with substitution lemmas.

It is supported over three main ingredients:
\begin{enumerate}
\item nameless (de Bruijn) syntax ;
\item parallel substitutions ;
\item explicit substitutions for automation.
\end{enumerate}

\vspace{2em} \hrule \vspace{2em}

Taking the naive example of an inductive definition of the $\Lam$-terms in \textit{Rocq}, we now display a definition using \textit{Autosubst}.

\begin{lstlisting}[language=Coq]
  Inductive term: Type :=
  | Var (x: var )
  | Lam (t: {bind term} )
  | App (s: term ) (t: term ) .
\end{lstlisting}

Here, the annotation \lst ${bind term}$ is an alias of the type \lst$term$.
We write this annotation in order to mark our binders in the syntax we want to formalise.  

This way, we may invoke the \textit{Autosubst} classes, automatically deriving the desired instances.

\begin{lstlisting}[language=Coq]
  Instance Ids_term : Ids term. derive. Defined.
  Instance Rename_term : Rename term. derive. Defined.
  Instance Subst_term : Subst term. derive. Defined.
  Instance SubstLemmas_term : SubstLemmas term. derive. Defined.
\end{lstlisting}

The first three lines derive the operations necessary to define the (parallel) substitution over a term.
\begin{enumerate}
\item Defining the function that maps every index into the corresponding variable term ($i \mapsto $ \lst$(Var i)$).
\item Defining the recursive function that instantiates a variable renaming over a term.
\item Defining the recursive function that instantiates a parallel substitution over a term (using the already defined renamings).
\end{enumerate}

Finally, there is also the proof of the substitution lemmas. 
Here, we see the power of this library: this process is done automatically, using the provided \lst$derive$ tactic.

\subsection{Mechanising $\Lam$-calculus}

We define the one step $\beta$-reduction altogether with the compatibility steps:
\begin{lstlisting}[language=Coq]
  Inductive step : relation term :=
  | Step_Beta s s' t : s' = s.[t .: ids] ->
                       step (App (Lam s) t) s'
  | Step_Abs s s' : step s s' ->
                    step (Lam s) (Lam s')
  | Step_App1 s s' t: step s s' ->
                      step (App s t) (App s' t)
  | Step_App2 s t t': step t t' ->
                      step (App s t) (App s t').
\end{lstlisting}

Formalising the typing system:

\begin{lstlisting}[language=Coq]
  Inductive sequent (gamma: var->type) : term -> type -> Prop := 
  | Ax (x: var) (A: type) :
    gamma x = A -> sequent gamma (Var x) A
  | Intro (t: term) (A B: type) :
    sequent (A .:gamma) t B -> sequent gamma (Lam t) (Arr A B)
  | Elim (s t: term) (A B: type) :
    sequent gamma s (Arr A B) -> sequent gamma t A -> sequent gamma (App s t) B.
\end{lstlisting}

The typing system definition is based on \cite{AutosubstManual}.

% \item Contextos infinitos
% \item Regras de tipificacao diferentes como em \cite{AutosubstManual}

%%% Local Variables:
%%% mode: LaTeX
%%% TeX-master: "../dissertation"
%%% End:
