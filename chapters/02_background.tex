\chapter{Background}
\label{c:background}

% In this chapter, we intend to take the example of formalising some theory about the $\lambda$-calculus as a motivation to display some core aspects involved in the more elaborate formalisations ahead.

The following chapter introduces essential background to aid the reading of this dissertation.
% It will take the scheme used in the subsequent chapters:
% a first part of theoretical presentation and a second part of mechanisation.
First, we introduce the $\lambda$-calculus and \sout{some basic knowledge around it}.
Then, we delve into the theory on mechanisation of meta-theory, specifically in the context of our work.
These concepts are introduced and motivated by the task of formalising the $\lambda$-calculus system introduced.

\section{Lambda Calculus}

\subsection{Syntax}

\cite{Hindley1997} \cite{Barendregt1987}

\begin{definition}[$\lambda$-terms]
  The $\lambda$-terms are defined by the following grammar:

  \[ M, N \ ::= \ x \ | \ (\lambda x . M) \ | \ (M N) \]

  where $x$ denotes any variable, typically in the range of $x, y, z$.
\end{definition}

\begin{notation}
  We will assume the usual notation conventions on $\lambda$-terms:

  \begin{enumerate}
  \item Outermost parenthesis are omitted.
  \item Multiple abstractions can be abreviated as $\lambda x y z . M$ instead of  $\lambda x . (\lambda y . (\lambda z . M))$.
  \item Multiple applications can be abreviated as $M N_1 N_2$ instead of $(M N_1) N_2$.
  \end{enumerate}
\end{notation}

\begin{definition}[Free variables]
  For every $\lambda$-term $M$, we recursively define the set of free variables in $M$, $FV(M)$, as follows:  
  \begin{align*}
    & FV( x ) = \{ x \}, \\
    & FV( \lambda x . M ) = FV(M) - \{ x \}, \\
    & FV( M N ) = FV(M) \cup FV(N).
  \end{align*}

  When a variable occurring in a term is not free it is said to be bound.
\end{definition}

% \begin{remark}
%   Informally, abstractions will behave as functions.
%   As so, we do not care about the names of bound variables. 
%   This idea is formally introduced next.
% \end{remark}

\begin{definition}[$\alpha$-equality]
  We say that two $\lambda$-terms are $\alpha$-equal when they only differ in the name of their bound variables.
\end{definition}

\begin{remark}
  The previous informal definition lets us take advantage of a variable naming convention.
  With this notion of $\alpha$-equality, the definition of substitution over $\lambda$-terms and meta-discussion of our syntax will be simplified.
  % After defining the substitution operation we may introduce a better and more formal definition of the $\alpha$-conversion.
  After defining the substitution operation we will rigorously introduce the definition for $\alpha$-conversion.
\end{remark}

\begin{convention} 
  We will use the \textit{variable convention} introduced in \cite{Barendregt1987}. Every $\lambda$-term that we refer from now on is chosen (via $\alpha$-equality) to have bound variables with different names from free variables.
\end{convention}

\begin{definition}[Substitution]
  For every $\lambda$-term $M$, we recursively define the substitution of the free variable $x$ by $N$ in $M$, $M[x := N]$, as follows:
  \begin{align*}
    & x[x := N] = N; \\
    & y[x := N] = y \text{, with } x \neq y; \\
    & (\lambda y . M_1)[x := N] = \lambda y . (M_1[x := N]); \\
    & (M_1 M_2)[x := N] = (M_1[x := N]) (M_2[x := N]).
  \end{align*}
\end{definition}

\begin{remark}
  Is is important to notice that by variable convention, the substitution operation described is capture-avoiding - bound variables will not be substituted ($x \in FV(M)$) and the free variables in $N$ will not be affected by the binders in $M$, as they are chosen to have different names. 
\end{remark}

\begin{definition}[Compatible Relation]
  We say that a binary relation on $\lambda$-terms, $R$, is compatible if it satisfies:
  \[
    \begin{prooftree}
      \hypo{ (M_1, M_2) \in R }
      \infer1{ (\lambda x . M_1, \lambda x . M_2) \in R } 
    \end{prooftree}
    \qquad
    \begin{prooftree}
      \hypo{ (M_1, M_2) \in R }
      \infer1{ (N M_1, N M_2) \in R } 
    \end{prooftree}
    \qquad
    \begin{prooftree}
      \hypo{ (M_1, M_2) \in R }
      \infer1{ (M_1 N, M_2 N) \in R } 
    \end{prooftree}
  \]
\end{definition}

\begin{definition}[$\alpha$-conversion]
  Consider the following binary relation on $\lambda$-terms:
  
  \[
    \lambda x . M =_\alpha \lambda y . M[x := y]. 
    \label{eq:alpha} \tag{$\alpha$}
  \]
  
  We define the $\alpha$-conversion as the least compatible, reflexive and transitive relation that satisfies \eqref{eq:alpha}.
\end{definition}

\begin{definition}[$\beta$-reduction]
  Consider the following binary relation on $\lambda$-terms:
  
  \[
    (\lambda x . M) N \to_\beta M[x := N]. \label{eq:beta} \tag{$\beta$}
  \]
  
  We define the $\beta$-reduction as the least compatible relation that satisfies \eqref{eq:beta}.
\end{definition}


\subsection{Types}

\cite{Barendregt2013}

\begin{definition}[Simple Types]
  The simple types are defined by the following grammar:
  
  \[
    A, B, C ::= p \ | \ (A \supset B)
  \]
  
  where $p$ denotes any atomic variable, typically in the range of $p, q, r$.
\end{definition}

\begin{notation}
  We will assume the usual notation conventions on simple types:  
  \begin{enumerate}
  \item Outermost parenthesis are omitted.
  \item Types associate to the right. Therefore, the type $A \supset (B \supset C)$ may often be written simply as $A \supset B \supset C$.
  \end{enumerate}
\end{notation}

\begin{definition}[Context]
  A context, $\Gamma, \Delta, \dots$, is a partial function from the variables of $\lambda$-terms to simple types.
\end{definition}

\begin{notation} \hfill
  \begin{enumerate}
  \item We may often think of the partial function of as the set of pairs $(x, A)$ written as $x:A$.
  \item We will also simplify the set notation of contexts as follows:
    \begin{align*}
      &\mapsto \{ \} \\
      x:A         &\mapsto \{ x:A \} \\
      x:A, \Gamma &\mapsto \{ x:A \} \cup \Gamma
    \end{align*}
  \end{enumerate}
\end{notation}

\begin{definition}[Typing Rules for $\lambda$-terms]
  A type-assignment or sequent is a triple, $\Gamma \vdash M:A$, that is inductively defined by the following inference rules (or typing rules):
  \[
    \begin{prooftree}
      \infer0[Var]{ x:A, \Gamma \vdash x:A } 
    \end{prooftree}
    \qquad
    \begin{prooftree}
      \hypo{ x:A, \Gamma \vdash M:B }
      \infer1[Abs]{ \Gamma \vdash \lambda x . M : A \supset B  } 
    \end{prooftree}
    \qquad
    \begin{prooftree}
      \hypo{ \Gamma \vdash M : A \supset B }
      \hypo{ \Gamma \vdash N : A }	
      \infer2[App]{ \Gamma \vdash M N : B } 
    \end{prooftree}
  \]
\end{definition}

% ---

\section{Mechanising Meta-theory in \textit{Coq}}

The formalisation we aim at is dependent on the theory provided by the \textit{Coq} proof assistant - the Calculus of Inductive Constructions.
We will follow assuming a basic knowledge on \textit{Coq} and its syntax to define inductive types and proof techniques.

\subsection{How to Denote Variables?}

% If we were to formalise such a system like the $\lambda$-calculus introduced above, we would probably create an inductive type like the following, in \textit{Coq}.
Formalising the $\lambda$-calculus in \textit{Coq} would generate an inductive definition similar to:

\begin{lstlisting}[language=Coq]
  Inductive term : Type :=
  | Var (x: var)
  | Lam (x: var) (t: term)
  | App (s: term) (t: term).
\end{lstlisting}

The question that every similar definition imposes is the definition of the $var$ type. Following the usual pen and paper approach, this type would be a subset of a string type, where a variable is just a placeholder for a name.

Of course this is fine when dealing with pen and paper proofs and definitions. To simplify this, we can even take advantage of conventions, like the one referenced above (by Barendregt). 
However, this variable definition can get rather exhausting  when it comes to rigorously define all this syntactical aspects and substitution operations.

There are several alternatives described in the literature of mechanisation of meta-theory. 
% This topic of binding was \sout{even} proposed in the POPLmark challenge \cite{POPLmark} as a way to discuss the potential of proof assistants.
The POPLmark challenge \cite{POPLmark} points to the topic of binding as a way to discuss the potential of modern proof assistants.

% Among the available solutions for the \textit{Coq} proof assistant, the \textit{AUTOSUBST} library .
The \textsc{Autosubst} library for the \textsc{Coq} proof assistant stood out as a great solution for our case. 
This library uses a combination of de Bruijn indices and explicit substitutions to \sout{tackle this "problem"}.

% POPl Mark
% falar de substituicoes ???

\subsection{De Bruijn Syntax}

\cite{deBruijn} \cite{AutosubstSchafer}

In the 1970s, de Bruijn started working on the \textsc{Automath} proof assistant and proposed a simplified syntax to deal with generic binders \cite{deBruijn}.
This approach is claimed to be good for meta-lingual discussion and for the computer and computer programme. In contrast, this syntax is further away from the human reader.

The main idea is to treat variables as indices (represented by natural numbers) and to interpret these indices as the distance to the respective binder.
\sout{Therefore, we will call these $\lambda$-terms the nameless $\lambda$-terms.}

\begin{definition}[nameless $\lambda$-terms]
  The nameless $\lambda$-terms are defined by the following grammar:

  \[ M, N \ ::= \ i \ | \ (\lambda . M) \ | \ (M N) \]

  where $i$ ranges over the natural numbers.
\end{definition}

\begin{remark}
  Nameless $\lambda$-terms have no $\alpha$-conversion since there is no freedom to choose the names of bound variables.
\end{remark}

\subsection{Explicit Parallel Substitutions}

\cite{AutosubstSchafer}

We refer to explicit substitutions \cite{Abadi} when the substitution operation is part of the syntax calculus.
Often, substitutions are dealt at a meta level.
However, a definition of capture-avoiding substitution over nameless $\lambda$-terms will require some manipulations over these explicit objects \cite{deBruijn}.

Having a reserved syntax for substitutions motivates us to generalise this notion.
Previously, the substitution object was a term that was going to take the place of some free variable.
A parallel substitution generalizes this concept to replace every free variable. 

Now, we present a calculus with explicit parallel substitutions.
We can see these explicit substitutions as sequences of terms, $\sigma = (M_i)_{i \in \mathbb{N}}$, where the term $M_i$ will replace the variable with index $i$.

\begin{definition}[$\lambda \sigma$-terms]
  The (nameless) $\lambda \sigma$-terms are defined by the following grammar:  
  \begin{align*}
    M, N         &::= \ i \ | \ (\lambda . M) \ | \ (M N) \ | \ M[\sigma] \\
    \sigma, \tau &::= \ id \ | \ \uparrow \ | \ M \cdot \sigma \ | \ \sigma \circ \tau
  \end{align*}
\end{definition}

% \begin{definition}[Rewriting System of $\sigma$-calculus]
\begin{definition}[Reductions over substitutions]

\end{definition}

\subsection{The Autosubst Library}
\cite{AutosubstSchafer}

The \textsc{Autosubst} library for the \textsc{Coq} proof assistant simplifies the formalisation of syntax with binders.
It provides tactics to define substitutions over an inductive definition of a nameless syntax.
Furthermore, it also has some automation available for proofs dealing with substitution lemmas and similar rewritings (?).

Taking the naive example of an inductive definition of the $\lambda$-terms in \textsc{Coq}, we now display a nameless definition of using \textsc{Autosubst}.

\begin{lstlisting}[language=Coq]
  Inductive term: Type :=
  | Var (x: var )
  | Lam (t: {bind term} )
  | App (s: term ) (t: term ) .
\end{lstlisting}

Now, to invoke the \textsc{Autosubst} definitions, we need the following lines.

\begin{lstlisting}[language=Coq]
  Instance Ids_term : Ids term. derive. Defined.
  Instance Rename_term : Rename term. derive. Defined.
  Instance Subst_term : Subst term. derive. Defined.
  Instance SubstLemmas_term : SubstLemmas term. derive. Defined.
\end{lstlisting}

These lines derive the instantiation of a substitution over a term.
\begin{enumerate}
\item Defining the function that maps every index into the corresponding variable term ($i \mapsto (Var \ i)$).
\item Defining the recursive function that instantiates a variable renaming over a term.
\item Defining the recursive function that instantiates a parallel substitution over a term (using the already defined renamings).
\end{enumerate}

Finally, there is also the proof of the substitution lemmas. 
Here, we see the power of this library: this process is done automatically, using the defined $derive$ tactic.

%%% Local Variables:
%%% mode: LaTeX
%%% TeX-master: "../dissertation"
%%% End:
